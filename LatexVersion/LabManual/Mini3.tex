\openepigraph{The simple act of paying attention can take you a long way.
}{---Keanu Reeves}


In the second mini-project, you will read, summarize and discuss the paper by \citeauthor{raz_suggestion_2006} (2013)\cite{raz_suggestion_2006}. This paper provides some background about the Stroop effect, which is a classic measure of selective attention, and then shows one manipulation that effectively helps people overcome Stroop interference. Your task in this lab will be to replicate the Stroop effect in one condition, and then attempt to change the size of the effect (make larger or smaller) in another condition.

\section{What's in store}

\begin{enumerate}
\item Students read paper and write QALMRI (15-20)
\item	Group discussion about paper (15-20)
\item	Students instructed their task is create their own Stroop design and employ a manipulation that increases or decreases the size of the Stroop effect
\item	Students break into groups and conduct a Stroop experiment, measuring the size of the Stroop effect in a "normal" condition, and in their manipulated condition.
\item	Groups analyze conduct a 2x2 ANOVA to see if their interaction was significant
\end{enumerate}

\section{Some background on the Stroop effect}

The Stroop effect is a well-known and classic phenomena in experimental psychology. The effect was first reported by J. R. Stroop (1935). Several hundreds of Stroop experiments have been conducted since 1935 and these experiments are summarized in Macleod's (1992) review.\cite{Stroop1935,macleod_half_1991}

The original Stroop task involved color naming of word stimuli that are printed in different ink colors. There are two important conditions involving congruent and incongruent item types. \emph{Congruent} items occur when the ink-color of the word matches the name of the word (e.g., the word red printed in red ink: \textcolor{red}{RED}). Incongruent items occur when the ink-color of the word does not match the name of the word (e.g., the word blue printed in red ink: \textcolor{red}{BLUE}). For each of these stimuli the task is to identify the ink-color of the word and ignore the written meaning of the word. Here are a few more examples of congruent and incongruent Stroop items:

Congruent: 	\textcolor{blue}{BLUE}, \textcolor{green}{GREEN}, \textcolor{yellow}{YELLOW}, \textcolor{red}{RED}

Incongruent:	\textcolor{red}{BLUE}, \textcolor{blue}{GREEN}, \textcolor{red}{YELLOW}, \textcolor{green}{RED} 

In the original set of experiments participants were presented with long lists of items and asked to read through the lists naming only the ink-colors and ignoring the word information. The important finding was that people were faster to finish reading lists that were composed of congruent items than incongruent items. This difference in reading time is termed the Stroop effect. For example, let’s say you were given a list of 60 congruent words and you read all of the ink-color names in 71 seconds. Next, you receive a list of 60 incongruent words and it takes you 94 seconds to read all of the ink-color names. You would compute your Stroop effect by taking the difference between the Congruent and Incongruent reading times. Congruent reading times are subtracted from incongruent reading times to give a positive Stroop effect value:

Stroop effect = Incongruent – Congruent  = 94 seconds – 71 seconds =  23 seconds

The Stroop test can be administered in many different ways. A common modern variant of the task is to present a single Stroop item on a computer screen per trial and have participants identify the ink-color by pressing a key or typing out the response. Several trials could be presented over the course of an experimental session, and the experimenter could vary whether upcoming trials are congruent or incongruent. The trial-based version of the Stroop procedure provides more experimental control and more precise measurement of reaction times.  This provides a more fine-grained measure of the time taken to respond congruent and incongruent items. Stroop effects using this procedure are usually measured in the milliseconds (ms). For example, reaction times for congruent items are usually around 500 ms, and reaction times for incongruent items are usually around 600 ms. So, the overall Stroop effect might be around 100 ms. 

\subsection{Stroop and Selective Attention}

The Stroop effect has been used as a tool to study various aspects of learning and attention. For example the Stroop effect has been used as a tool to study cognitive control. Cognitive control refers broadly to the psychological processes that allow people to plan, coordinate, and execute actions necessary to accomplish goals. A central aspect of cognitive control involves the attention processes that are responsible for selecting task-relevant information and ignoring task-irrelevant information when performing a task. 

The Stroop procedure provides a simple, yet effective, method for presenting people with two sources of information, one that is task-relevant (color) and one that is task-irrelevant (word). For this reason, Stroop items are sometimes referred to as bi-valent stimuli, in that they present two sources of information. Congruent Stroop items (blue in blue) do not present much of a selective attention challenge, both the word and the color information point to the same response. Incongruent items (blue in red) do present a selective attention challenge, the color points to the correct response and the word points to an incorrect response. When faced with incongruent items people must pay attention to the relevant color information and ignore or somehow filter out the irrelevant word information. The fact that people get Strooped, or that the Stroop effect exists at all, tells us something important about selective attention in this task. Specifically, people can not completely ignore the irrelevant word information. If people could completely ignore the irrelevant word information then the Stroop effect would cease to appear. People would be just as fast identifying congruent and incongruent items because they would be able to successfully ignore word information. 

Using the logic above the Stroop effect is often taken as a measure of selective attention ability. This means that changes in the size of the Stroop effect may represent differences in selective attention ability. People who have very large Stroop effects have poor attentional control over their ability to ignore the word stimulus. People who have very small Stroop effects have excellent attentional control over their ability to ignore the word stimulus. 




