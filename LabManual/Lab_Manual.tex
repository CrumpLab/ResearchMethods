\documentclass[oneside]{tufte-book}
\usepackage{listings}
\usepackage{color}
\usepackage{textcomp}
\usepackage{graphicx}
\usepackage{enumerate}
\usepackage{hyperref}
\usepackage{makeidx}
\usepackage{natbib}
\usepackage{amsmath}
\usepackage{pdfpages}

\hypersetup{colorlinks=true}

\usepackage{microtype}
\usepackage{fancyvrb} % Allows customization of verbatim environments
\fvset{fontsize=\normalsize} % The font size of all verbatim text can be changed here

\newcommand{\monthyear}{\ifcase\month\or January\or February\or March\or April\or May\or June\or July\or August\or September\or October\or November\or December\fi\space\number\year}



\newcommand{\VerbBar}{|}
\newcommand{\VERB}{\Verb[commandchars=\\\{\}]}
\DefineVerbatimEnvironment{Highlighting}{Verbatim}{commandchars=\\\{\}}
% Add ',fontsize=\small' for more characters per line
\usepackage{framed}
\definecolor{shadecolor}{RGB}{248,248,248}
\newenvironment{Shaded}{\begin{snugshade}}{\end{snugshade}}
\newcommand{\KeywordTok}[1]{\textcolor[rgb]{0.13,0.29,0.53}{\textbf{{#1}}}}
\newcommand{\DataTypeTok}[1]{\textcolor[rgb]{0.13,0.29,0.53}{{#1}}}
\newcommand{\DecValTok}[1]{\textcolor[rgb]{0.00,0.00,0.81}{{#1}}}
\newcommand{\BaseNTok}[1]{\textcolor[rgb]{0.00,0.00,0.81}{{#1}}}
\newcommand{\FloatTok}[1]{\textcolor[rgb]{0.00,0.00,0.81}{{#1}}}
\newcommand{\ConstantTok}[1]{\textcolor[rgb]{0.00,0.00,0.00}{{#1}}}
\newcommand{\CharTok}[1]{\textcolor[rgb]{0.31,0.60,0.02}{{#1}}}
\newcommand{\SpecialCharTok}[1]{\textcolor[rgb]{0.00,0.00,0.00}{{#1}}}
\newcommand{\StringTok}[1]{\textcolor[rgb]{0.31,0.60,0.02}{{#1}}}
\newcommand{\VerbatimStringTok}[1]{\textcolor[rgb]{0.31,0.60,0.02}{{#1}}}
\newcommand{\SpecialStringTok}[1]{\textcolor[rgb]{0.31,0.60,0.02}{{#1}}}
\newcommand{\ImportTok}[1]{{#1}}
\newcommand{\CommentTok}[1]{\textcolor[rgb]{0.56,0.35,0.01}{\textit{{#1}}}}
\newcommand{\DocumentationTok}[1]{\textcolor[rgb]{0.56,0.35,0.01}{\textbf{\textit{{#1}}}}}
\newcommand{\AnnotationTok}[1]{\textcolor[rgb]{0.56,0.35,0.01}{\textbf{\textit{{#1}}}}}
\newcommand{\CommentVarTok}[1]{\textcolor[rgb]{0.56,0.35,0.01}{\textbf{\textit{{#1}}}}}
\newcommand{\OtherTok}[1]{\textcolor[rgb]{0.56,0.35,0.01}{{#1}}}
\newcommand{\FunctionTok}[1]{\textcolor[rgb]{0.00,0.00,0.00}{{#1}}}
\newcommand{\VariableTok}[1]{\textcolor[rgb]{0.00,0.00,0.00}{{#1}}}
\newcommand{\ControlFlowTok}[1]{\textcolor[rgb]{0.13,0.29,0.53}{\textbf{{#1}}}}
\newcommand{\OperatorTok}[1]{\textcolor[rgb]{0.81,0.36,0.00}{\textbf{{#1}}}}
\newcommand{\BuiltInTok}[1]{{#1}}
\newcommand{\ExtensionTok}[1]{{#1}}
\newcommand{\PreprocessorTok}[1]{\textcolor[rgb]{0.56,0.35,0.01}{\textit{{#1}}}}
\newcommand{\AttributeTok}[1]{\textcolor[rgb]{0.77,0.63,0.00}{{#1}}}
\newcommand{\RegionMarkerTok}[1]{{#1}}
\newcommand{\InformationTok}[1]{\textcolor[rgb]{0.56,0.35,0.01}{\textbf{\textit{{#1}}}}}
\newcommand{\WarningTok}[1]{\textcolor[rgb]{0.56,0.35,0.01}{\textbf{\textit{{#1}}}}}
\newcommand{\AlertTok}[1]{\textcolor[rgb]{0.94,0.16,0.16}{{#1}}}
\newcommand{\ErrorTok}[1]{\textcolor[rgb]{0.64,0.00,0.00}{\textbf{{#1}}}}
\newcommand{\NormalTok}[1]{{#1}}





\newcommand{\openepigraph}[2]{ % This block sets up a command for printing an epigraph with 2 arguments - the quote and the author
\begin{fullwidth}
\sffamily\large
\begin{doublespace}
\noindent\allcaps{#1}\\ % The quote
\noindent\allcaps{#2} % The author
\end{doublespace}
\end{fullwidth}
}




\definecolor{shadecolor}{rgb}{.97, .97, .97}
\lstset{ %
  language=R,                     % the language of the code
  basicstyle=\footnotesize,       % the size of the fonts that are used for the code
  numbers=left,                   % where to put the line-numbers
  numberstyle=\tiny\color{black},  % the style that is used for the line-numbers
  stepnumber=1,                   % the step between two line-numbers. If it's 1, each line
                                  % will be numbered
  numbersep=5pt,                  % how far the line-numbers are from the code
  backgroundcolor=\color{shadecolor},  % choose the background color. You must add \usepackage{color}
  showspaces=false,               % show spaces adding particular underscores
  showstringspaces=false,         % underline spaces within strings
  showtabs=false,                 % show tabs within strings adding particular underscores
                     % adds a frame around the code
  rulecolor=\color{black},        % if not set, the frame-color may be changed on line-breaks within not-black text (e.g. commens (green here))
  tabsize=2,                      % sets default tabsize to 2 spaces
  captionpos=b,                   % sets the caption-position to bottom
  breaklines=true,                % sets automatic line breaking
  breakatwhitespace=false,        % sets if automatic breaks should only happen at whitespace
  title=\lstname,                 % show the filename of files included with \lstinputlisting;
                                  % also try caption instead of title
  keywordstyle=\color{blue},      % keyword style
  commentstyle=\color{dkgreen},   % comment style
  stringstyle=\color{red},      % string literal style
  escapeinside={\%*}{*)},         % if you want to add a comment within your code
  morekeywords={*,...},            % if you want to add more keywords to the set
  basicstyle=\ttfamily
} 

\setcounter{tocdepth}{4}


\makeatletter
\def\maxwidth{ %
  \ifdim\Gin@nat@width>\linewidth
    \linewidth
  \else
    \Gin@nat@width
  \fi
}
\makeatother

\definecolor{fgcolor}{rgb}{0.345, 0.345, 0.345}
\newcommand{\hlnum}[1]{\textcolor[rgb]{0.686,0.059,0.569}{#1}}%
\newcommand{\hlstr}[1]{\textcolor[rgb]{0.192,0.494,0.8}{#1}}%
\newcommand{\hlcom}[1]{\textcolor[rgb]{0.678,0.584,0.686}{\textit{#1}}}%
\newcommand{\hlopt}[1]{\textcolor[rgb]{0,0,0}{#1}}%
\newcommand{\hlstd}[1]{\textcolor[rgb]{0.345,0.345,0.345}{#1}}%
\newcommand{\hlkwa}[1]{\textcolor[rgb]{0.161,0.373,0.58}{\textbf{#1}}}%
\newcommand{\hlkwb}[1]{\textcolor[rgb]{0.69,0.353,0.396}{#1}}%
\newcommand{\hlkwc}[1]{\textcolor[rgb]{0.333,0.667,0.333}{#1}}%
\newcommand{\hlkwd}[1]{\textcolor[rgb]{0.737,0.353,0.396}{\textbf{#1}}}%

\usepackage{framed}
\makeatletter
\newenvironment{kframe}{%
 \def\at@end@of@kframe{}%
 \ifinner\ifhmode%
  \def\at@end@of@kframe{\end{minipage}}%
  \begin{minipage}{\columnwidth}%
 \fi\fi%
 \def\FrameCommand##1{\hskip\@totalleftmargin \hskip-\fboxsep
 \colorbox{shadecolor}{##1}\hskip-\fboxsep
     % There is no \\@totalrightmargin, so:
     \hskip-\linewidth \hskip-\@totalleftmargin \hskip\columnwidth}%
 \MakeFramed {\advance\hsize-\width
   \@totalleftmargin\z@ \linewidth\hsize
   \@setminipage}}%
 {\par\unskip\endMakeFramed%
 \at@end@of@kframe}
\makeatother

\definecolor{shadecolor}{rgb}{.97, .97, .97}
\definecolor{messagecolor}{rgb}{0, 0, 0}
\definecolor{warningcolor}{rgb}{1, 0, 1}
\definecolor{errorcolor}{rgb}{1, 0, 0}
\newenvironment{knitrout}{}{} % an empty environment to be redefined in TeX

\usepackage{alltt}
\IfFileExists{upquote.sty}{\usepackage{upquote}}{}



%opening
%\title{Research Methods in Psychology}

\title{Experimental Psychology Lab Manual Fall 2017}

\author{Matthew Crump}
\publisher{\textit{CURRENT DRAFT VERSION 1, \today}}
\makeindex
\begin{document}

\frontmatter

\maketitle

%----------------------------------------------------------------------------------------
%	EPIGRAPH
%----------------------------------------------------------------------------------------

\thispagestyle{empty}

\openepigraph{Children are born true scientists. They spontaneously experiment and experience and re-experience again. They select, combine, and test, seeking to find order in their experiences - "which is the mostest? which is the leastest?" They smell, taste, bite, and touch-test for hardness, softness, springiness, roughness, smoothness, coldness, warmness: they heft, shake, punch, squeeze, push, crush, rub, and try to pull things apart.}{- Buckminster Fuller}


%----------------------------------------------------------------------------------------
%	COPYRIGHT PAGE
%----------------------------------------------------------------------------------------

\tableofcontents




%\chapter{The Workshop of Psychological Science}

%\input{Roadmap.tex}


%\chapter{Experiment Number One}
%\input{Chapter1.tex}

\chapter{1 Overview}
\lhead{\allcaps{Overview}}

\openepigraph{Today was good. Today was fun. Tomorrow is another one.}{---Dr. Seuss}

\section{General Overview}

The weekly labs are structured to give students experience with conducting experiments, analyzing data, thinking critically about theory and data, and communicating their results and analysis in writing and oral presentation. 

Lab instructors facilitate this process by guiding students through the demands of each of the labs. Lab instructors are there to help you, so ask them questions when you need answers!

\subsection{Lab structure: Major and mini projects, and presentations}

The entire semester is divided into three major projects, and 3 mini projects. They are ordered consecutively to build upon skills. For example, the first projects focus on data analysis skills involving t-tests, then one-way ANOVAs, and finally 2x2 Factorial designs. Throughout each of the labs you will learn how to create experiments and conducts analyses on the data. Then you will employ these skills in the last half of the semester when you form groups and conduct, analyze, report and present an experiment of your own design.


\subsection{Major projects}  

Each major project involves students completing an experiment as a class (using themselves as subjects to collect data). Students will learn about the conceptual issues behind the experiment, collect data on themselves, and analyze the class data using appropriate statistics. Each student will then be responsible for writing a short (5+ page) APA style report about the project. 

The first two major projects involve predefined experiments that the class completes. These two projects are roughly finished mid-semester, and are intended to train students in the skills needed to complete the final project. The last major project is the final project, where students form groups and complete an experiment based on their own design. 

Each lab instructor is responsible for grading each of their students papers. Individual lab instuctors will explain to their sections how their grading scheme will work.
 
\subsection{Mini projects} 
In the first half of the semester, each of the weeks that does not introduce a new major project are reserved for mini-projects. These projects are intended to be completed within one lab session. Each of the mini-projects involve 1) reading and understanding a primary source, and 2) attempting to replicate the result in the paper. These are graded on a pass/fail basis, where a pass is given to students who show up and participate in the lab (regardless of whether their experiment turns out.)

\subsection{Presentations} 
The final project involves two presentations. An individual presentation, and a group presentation. Prior to forming groups for the final project, each student will give a short (2-3) minute pitch for their project idea. Then, students will form groups (choose one of the members project ideas, or generate a new one) and begin working on their final project. The last lab is reserved for the group presentations where each group gives a 10 minute research presentation. 

\subsection{Grading}

Each lab instructor will grade the work of the students in their sections. See the course syllabus for information on how each of the lab components weigh into the final grade for the course.

\section{Lab Resources}

There are many resources to help you complete the lab assignments. These are included in this lab manual, as well as online.

\subsection{Website}

As much as possible all of the information for this course will be posted on the course website:

\url{http://crumplab.github.io/courses/experimental/}

\subsection{Lab rooms} 

There should be one computer per student in each of the lab rooms. These computers should have SPSS, Excel, Office, R, Superlab, LIVECODE, and Psychopy installed on them. 

\subsection{Lab manual} 

This lab manual contains instructions for each the lab assignments, as well as helpful tutorials for learning skills to analyze and report data.

\section{Lab Schedule}

\subsection{Lab 1: Overview}
\begin{enumerate}
\item Meet and greet your instructor and fellow classmates, and learn about what is in store for the labs this semester	

a.	You will write three APA style research reports. Papers 1 and 2 will be on predefined projects, and Paper 3 will be based on a final project where students form groups to complete an experiment of their own design

b.	The final project will involve two presentations, an individual presentation and a group presentation. 

c.	The first half of the semester involves completing Papers 1 and 2, and a few mini-projects that occur between papers 1 and 2. These will build the skills necessary to complete the final project which will take up most of the second half of the semester
\item	Students are expected to show up and be on time for labs
\item	Ask questions
%\item	Administer Confidence in Reading Primary sources Questionnaire
%
%a.	Explain that part of the lab and lecture curriculum is designed to help students build skills in reading primary source material, and that we will be conducting some research to assess these outcomes. 
%
%b.	Give verbal consent procedure
%
%c.	Hand out questionnaires (10 minutes), collect them and return them to Nick Brosowsky’s mailbox in the main office.
\item	Go over QALMRI method using provided QALMRI materials
\item	Discuss/Review the components of writing an APA paper
\end{enumerate}

\subsection{Lab 2: Paper project 1}

Students replicate the results of \citeauthor{song_if_2008} (2008)\cite{song_if_2008}.

\begin{enumerate}
\item Administer the experiment using the provided materials

a.	Students will receive a piece of paper with instructions. They will read the description of an exercise routine, and then answer the questions about what they read.
\item	Reading and understanding the primary source

a.	Students will be given the Song \& Schwarz (2008) paper, and the to-be-filled in QALMRI worksheet. They will be given 15-20 minutes to read the paper, and in small groups attempt to fill out the QALMRI worksheet for the paper

b.	Group discussion of the paper and the QALMRI
\item	Collect and analyze the data
\item	Discuss the paper assignment
\end{enumerate}

\subsection{Lab 3: Paper project 1 continued}
\begin{enumerate}
\item More time to work on the first paper. Use your lab instructor as a resource and ask questions if you need more info.

E.g., Review APA style, review the structure and content of the paper, review the results, edit each others work, etc.
\item Due dates are set by the lab instructor
\end{enumerate}

\subsection{Lab 4: MiniProject 1 Nairne, Pandeirada, \& Thompson (2008)}

Students replicate the results of \citeauthor{nairne_adaptive_2008} (2008)\cite{nairne_adaptive_2008}.

\begin{enumerate}
\item Students read paper and write QALMRI (15-20)
\item Group discussion about paper (15-20)
\item	Students attempt to replicate the major findings in the paper

a.	Break into four-five groups, each group assigned an encoding condition (Survival, Pleasantness, Imagery, Self-reference, Intentional learning)

b.	Each group picks their own 30 words. Can follow same procedure as in paper by choosing 30 words from Overschelde, Rawson, \& Dunlosky (2004) \cite{van_overschelde_category_2004}.

c.	Groups try to run at least 10 participants in their condition, recording proportion of correctly recalled words

d.	Groups enter their collected data into the master spreadsheet, which is given back to groups upon data completion

\item Discussion of how to analyze the data
\item Group attempt to analyze the data using t-tests and one-way ANOVA to determine if the survival framing produced better recall than the other conditions.
\end{enumerate}

\subsection{Lab 5: Mini Project 2 Stoet et al. (2013)}

Students replicate the results of Stoet et al. (2008) \cite{stoet_are_2013}.

\begin{enumerate}
\item Students read paper and write QALMRI (15-20)
\item Group discussion about paper (15-20)
\item Students download the task-switching program available from the website and individually complete the task. Individual students then enter their data in the master spreadsheet
\item Discussion of how to analyze the data. Major analysis goals are:

a.	Was there a mixing cost? Compare pure lists to mixed lists

b.	Was there a switching cost? Compare switch vs. repeat trials in pure lists

c.	Was there a gender effect?

\item Students break into groups to analyze the data.
\end{enumerate}

\subsection{Lab 6: Mini Project 4 Raz et al. (2006)}
Students replicate the results of Raz et al. (2006) \cite{raz_suggestion_2006}
\begin{enumerate}
\item Students read paper and write QALMRI (15-20)
\item	Group discussion about paper (15-20)
\item	Students instructed their task is create their own Stroop design and employ a manipulation that increases or decreases the size of the Stroop effect
\item	Students break into groups and conduct a Stroop experiment, measuring the size of the Stroop effect in a "normal" condition, and in their manipulated condition.
\item	Groups analyze conduct a 2x2 ANOVA to see if their interaction was significant
\end{enumerate}


\subsection{Lab 7 : Paper Project 2 Yin (1969)}
Students replicate the results of Yin (1969) \cite{yin_looking_1969}
\begin{enumerate}
\item Students read paper and write QALMRI (15-20)
\item	Group discussion about paper (15-20)
\item	Students complete computerized task and report their data in the master spreadsheet. Data is given back to students for analysis
\item	Discussion about writing the 2nd paper
\end{enumerate}

\subsection{Lab 8: Paper project 2 continued}
\begin{enumerate}
\item Extra-time for completing second paper, again use your lab instructor as a resource, they are there to help.
\end{enumerate}


\subsection{Lab 9: Brainstorming for Final project}

\begin{enumerate}
\item Learn about details of the final project
\item Learn about details of the individual presentation (next lab)
\item Brainstorming session allowing students to think about possible projects that they would propose for their individual project
\end{enumerate}

\subsection{Lab 10: Individual Presentations}
\begin{enumerate}
\item Students give their individual (2-3 minute) presentations
\item Students are divided into groups for their final project
\item Each group decides on the experiment for their final project. This could be from one of the individual project ideas, or a new idea. Groups need permission from the lab instructor for their final project before data collection begins
\end{enumerate}

\subsection{Lab 11-13: Group work on Final Project}
\begin{enumerate}
\item Help groups finalize their final project aims
\item Groups collect data
\item Discuss requirements for final presentation and paper
\end{enumerate}

\subsection{Lab 14: Final group presentations}



\chapter{2 The QALMRI Method}
\lhead{\allcaps{The QALMRI Method}}

\openepigraph{Question Alternatives Logic Method Results Inference}{---The QALMRI Method}

The general goal of this course is to give you experience with the scientific process of asking and answering psychological questions. At the end of the course you should gain at least two kinds of skills. First, in the labs you will learn to ask and answer your own questions by conducting, analyzing, and reporting the results of experiments that you conduct. Second, you learn how understand the research literature, where other researchers have asked and answered questions, and communicated them publicly by publishing journal articles on their research.

There are a lot of details involved in asking and answering questions in science, and you can easily see these details on display by reading any published peer-review journal article. You will become very familiar these details as you read more papers, and as you write your own APA style research reports.

The details can be daunting, mainly because the format and structure of articles may be unfamiliar, and the language that scientists use is often very specific (or jargony). As a result, if you are not already an expert in a particular domain, it can be very difficult to read and understand research presented in journal articles. Furthermore, if you do not understand the concepts behind the questions, and the details behind the methods and results, then it is difficult to write your own research reports about experiments that you conduct. Have no fear, this lab manual is here to guide you through the process. 

Ultimately, by learning the skills we teach you throughout this semester, you will be able to critically evaluate your own research, as well as the research literature at large. This will allow you form your own opinions about the process of asking and answering questions. For example, when you learn how to critically evaluate research you will be able to:
 
\begin{enumerate}
\item Evaluate whether or not you should believe particular scientific claims, or claims made by the media about new "research findings"
\item Look at the evidence to see whether it actually provides an answer to the question that was being asked
\item Look at the questions to see if they are good ones, and learn how to ask better questions
\item Understand how theories and hypotheses work and make predictions about psychological phenomena
\item Learn how to find scientific research that has been conducted on topics of your own interest, and then evaluate the claims and evidence for yourself. For example, the skills you learn here about evaluating psychological research are the same skills that you would need to evaluate medical research. So, if you are wondering whether a particular medical procedure, or drug, or diet works or doesn't work, then you can read the research for yourself to arrive at your own informed opinion.
\end{enumerate}

\section{The QALMRI Method}

The first step is learning how to read journal articles. We will use the QALMRI method throughout this course as a tool to help you identify the major ideas and findings presented in journal articles. The major aspects of the QALMRI method will also help you identify whether your own research reports contain the information necessary to communicate your own research. First, we will explain the QALMRI method, then we will explain how you will be asked to use it throughout the course.

\marginnote{Adapted nearly verbatim from: Kosslyn, S.M. \& Rosenberg, R.S. (2001). Psychology: The Brain, The Person, The World. Boston: Allyn \& Bacon.}

The QALMRI method provides a means for critically evaluating experiments, as well as for organizing your own experiment proposals. It helps you to find connections between theory and data by making explicit the question being asked, the approach used to answer it, and the implications of the answer. QALMRI is an acronym, and each letter identifies critical parts of research articles.

\subsection{Q stands for Question}

All research begins with a question, and the point of the research is to answer it. For example, we can ask whether a placebo is better than no action in alleviating depression. For most journal articles, the General Introduction should tell the reader what question the article is addressing, and why it is important enough that anyone should care about the answer. Questions fall into two categories: broad and specific. In your QALMRI, state both the broad and the specific questions being asked. Broad questions are typically too general to answer in a single experiment, although one should view the experiment as one step on a journey to answer the broad question. An example of a broad question might be "Does language influence perception?" This sort of question provides the general topic of the paper, and can only be answered through compiling many experimental results. In contrast, the specific question can typically be addressed in a single experiment or set of experiments. A specific question might be "If one language has a specific term for one color, and another language does not have any term for that color, will speakers of the two languages perceive the color differently?" 

Again, be sure to identify the broad and specific question relevant to your data collection. 

\subsection{A stands for Alternatives}

Good experiments consider at least 2 possible alternative answers to a specific question, and explains why both answers are plausible. For example, the possibility that speakers of different languages will perceive colors differently is plausible based on evidence that top-down processes can affect perception. The alternative hypothesis, that language does not influence perception of color, is also plausible because color perception in particular might be impervious to top-down influences. That is, it might be based solely on properties of the visual system which are unaffected by language. When describing an existing article or when proposing an experiment, you should identify the alternatives the authors considered. There are always at least 2 alternatives: that factor X will show an effect, or that it won't (that a null result will be obtained). If possible, identify other alternative patterns as well. 

\subsection{L stands for Logic}

The logic of the study identifies how the experiment's design will allow the experimenter to distinguish among the alternatives. The logic is typically explained towards the end of the study's introduction, and has the following structure: If alternative 1 (and not alternative 2) is correct, then when a particular variable is manipulated, the participants' behavior should change in a certain way. For example, the logic of the color experiment would be: If a person's native language influences their perception of color, then speakers who have a term for a given color should respond differently to that color than speakers whose language contains no term for that color. Alternatively, if language does not influence color perception, then speakers who have a color term should respond no differently than speakers who lack the term. Note that the logic of the experiment is integrally connected to the alternatives you stated in the last section. Indeed, this section should be comprised of a series of "If, then" statements in which you restate the alternatives you offered ("If X,"), and then state what pattern of data would support that alternative ("then Y"). You should therefore have equal numbers of alternatives and If…then statements. 

\subsection{M stands for Method}

This section identifies the procedures that will be used to implement the logical design. It should state the independent variable (the factor being experimentally manipulated) and the dependent variable (the behavior being measured) of the experiment. It should also describe the subjects, including whether subjects were divided into groups receiving different experimental manipulations. What materials were used to conduct the experiment, and what were the experimental stimuli like? 

\subsection{R stands for Results}

What was the outcome of the experiment? Describe the results of the primary measures of interest. For example, did different subject groups yield different group means? What were these means? Or did the entire subject population produce a distinctive pattern of responses? Describe that pattern. Did the results seem reliable, or do you feel they might have been an artifact of the way the experiment was conducted? For this section, it is often a good idea to use graphs or tables to illustrate the pattern of data you obtained. 

\subsection{I stands for Inferences}

What can the results of the experiment tell us about the alternatives? If the study was well designed, the results should allow you to eliminate at least one of the possible alternatives. For example, if a language lacks a color word but the speakers of that language respond to the color no differently than speakers of a language lacking a term for the color, then the experiment supports the view that language does not influence color perception. At this point, take a step back and think about any potential problems with the experiment that could have led to the pattern of results you obtained. Were there confounds that could have caused the results? For example, if you did find a difference between the subject groups, are there other ways in which the groups differ that are not language-related? Might this have caused the result? Were there problems during the data collection? In addition, this is the section in which to consider the hypothetical next step in answering the broad question. If you were to conduct a follow-up experiment, what would it be (hint: think of questions that remain unanswered by the present results, and sketch a study that could bear on one or more of those questions)? What questions do your results raise? 

\section{Writing a QALMRI}

Writing a QALMRI for any research paper (one that you are writing, or one that you are reading) is simply writing short answers to each of these questions using clear and concise language. It is a condensed, short-form, version of the research. To be even more specific, your task is to answer these questions:

\marginnote{\allcaps{How long is a QALMRI?} Long-enough to answer each question with clear and brief sentences.} 

\begin{itemize}
\item Question: What was the broad question? What was the specific question?
\item Alternative hypotheses: What were the hypotheses?
\item Logic: If hypothesis 1 was true, what was the predicted outcome? What was the predicted outcome if hypothesis 2 was true?
\item Method: What was the experimental design?
\item Results: What was the pattern of data?
\item Inferences: What can be concluded about the hypotheses based on the data? What can be concluded about the specific and broad question? What are the next steps?
\end{itemize}

In many of the weekly labs you will read a primary research article, and then write your own QALMRI summary. This will help you extract the big ideas and findings from the research. You will also discuss your QALMRIs as a group to make sure that everyone is on the same page about what the research was about, and what the research showed.

\section{Example QALMRI}

Even if you haven't read the article, reading a QALMRI should provide you with enough information to get a basic idea of what the article was about. The following QALMRI summarizes the article by Crump \& Logan (2010). \cite{crump_warning:_2010}

\subsection{What was the broad and specific question?}

The broad questions are about spatial cognition. How do people understand and represent the spatial relationships between objects in the environment? Do people have "spatial maps" in their head?

The specific question is how do typists know where the keys are on the QWERTY keyboard?

\subsection{What are the alternatives?}

\begin{enumerate}
\item Typists have an internal cognitive spatial map of the keyboard that they use to guide their fingers during typing
\item Typists do not have a map-like representation, instead they rely on learned associations between cues such as the feel of the keyboard to guide their fingers during typing
\end{enumerate}

\subsection{What is the logic?}
\begin{enumerate}
\item If typists have an internal map of the keyboard, then they should be able guide their fingers to correct locations based on the map alone, and no feedback from the environment. For example, if we could measure "air-typing" without a keyboard, then typists should still be able to put their fingers in the correct locations even when the keyboard is missing because they are relying on their internal map.

\item If typists do not use an internal map of the keyboard, then their finger movements should become slow and inaccurate when they try to type without a keyboard, or in other conditions that change the normal feel of the keyboard, and thereby remove the cues that typists use to direct their fingers.
\end{enumerate}

\subsection{What is the Method?}
Typists copied paragraphs in four conditions that manipulated tactile (touching) feedback from the keyboard. They typed on a normal keyboard, a keyboard with the keys removed exposing the rubber buttons underneath, a flat circuit board without, and on a flat table with a laser projection keyboard.

\subsection{Results}

Typists were fast and accurate in the normal keyboard condition. Typists were slow and inaccurate in all of the other keyboard conditions, where normal tactile feedback was removed.

\subsection{Inference}

The results are \emph{not consistent} with the internal map hypothesis. If typists had an internal map, and did not rely on tactile cues, then they should have typed normally even when the cues were removed. The results are consistent with learned association hypothesis, that typists rely on cues, like the feel of the keyboard, that are associated with particular key locations.

\section{Finding the QALMRI sections in a journal article}

Most psychology journal articles are in APA format. See the textbook, and the sections about APA style in this lab manual for more on that format. APA format makes it relatively easy to find the sections of the QALMRI. The basic parts of an APA manuscript are 
\begin{enumerate}
\item Title - Offers clues about the big idea and finding
\item Abstract - Usually a useful summary of the whole research.
\item Introduction - Big questions are usually near the beginning, specific questions are usually in the middle. Alternatives (or Hypotheses) and Logic are usually just before the Experiment, and sometimes in the short introduction to the experiment.
\item Methods - a whole section devoted to the M part
\item Results - a whole section devoted to the R part
\item Discussion - Usually provides a summary of the main important finding right at the beginning
\item General Discussion - Should connect the main finding to the hypotheses and questions under investigation
\end{enumerate}

Most articles that are not in APA style also have very similar sections, and you should find the answers to the QALMRI in the same places. Some journals like Science and Nature have the method sections at the very end. Also, some journals have short article formats, where the research is presented in one or two pages. Oftentimes short reports do not have headers; nevertheless, the answers to the QAL parts are usually near the beginning and middle, and the answers to the RMI parts are usually near the end. 

To get some practice extracting QALMRI from a journal article, let's look at short report published in Science. We'll look at the Mehl et al. (2007) \cite{mehl_are_2007} paper, discussed in chapter 1 of the textbook. It's only one page long, so it's an easy read.

 \href{https://login.ez-proxy.brooklyn.cuny.edu/login?url=http://science.sciencemag.org/content/317/5834/82}{Click this link to get the paper online}. Note, if you are on the Brooklyn College network, this should take you directly to the paper. If you are at home, you will need to first sign into the library.

\subsection{Finding the Big Question}

You might say the big question is implied by first sentence, "Sex differences in conversational behavior have long been a topic of public and scientific interest". The big question is not stated directly, but you could infer that the questions are something like, 1) Are there sex differences in talking, or other conversational behavior?, 2) Do women talk more than men?, 3) Why do women talk more than men?, and so on. 

You might also read this article and conclude that the authors did not really spell out the big questions. After all, they only had a single page to communicate their research. So, it is important to realize that there are big questions that may not actually be stated. You, the reader, need to connect the dots. For example, we could say some of the really big questions are: 1) What are the main sex differences in communication, language, and cognition in general, 2) How do people use language to communicate, 3) What cognitive, biological, genetic, and environmental mechanisms cause sex differences in these behaviors? You can see that the research relates to these bigger questions, even if they are not explicitly stated.

\subsection{Finding the Specific question}

In this case, the specific question is hard to miss. It's in the title, "Are women really more talkative than men?". The authors motivate asking this question by pointing out that systematic research has not actually measured how many words men and women say on average per day.

\subsection{Alternatives}

The hypotheses in this article are pretty straightforward because they are clearly related to an empirical question. One hypothesis is alluded to in the second sentence, where the authors claim that western society has a stereotype that women are more talkative than men. In this case, the hypotheses in question are not about psychological processes or mechanisms, they are simply about the data. The authors also do not formally state two hypotheses. We have to do that for them.

\begin{itemize}
\item Hypothesis 1: Women are more talkative than men
\item Hypothesis 2: Women are not more talkative than men
\end{itemize}

\subsection{Logic}

In this case, the logic of the hypotheses is nearly identical to the hypotheses themselves, but simply stated in terms of a logical if/then structure.

\begin{itemize}
\item If, women are more talkative than men, then women should on average produce more words per day than men
\item If, women are not more talkative than men, then women should on average produce the same or less number of words per day than men.
\end{itemize}

\subsection{Methods}

You can see the methods described in paragraphs three and four. Basically, they recorded subjects all day long in their natural environments, and then counted how many words they said.

\subsection{Results}

The results are in paragraph six, and presented in the table. You already know the question: "Do women talk more than men?", and that they measured the number of words spoken by groups of men and women, so you should expect to find some average data showing the actual counts. We see this in the weighted average in the table. Women said on average 16,215 words per day, and men said 15,669 words per day. The results section says that this result was not statistically different, and could have easily been produced by chance. You can even see that some of the sub-groups showed the opposite pattern of data, with men saying more words per day than women.

\subsection{Inference}

The authors make inferences about their hypotheses in the last two paragraphs. Their major inference is that, contrary to popular belief, women are not more talkative than men. They also discuss some limitations of their work.

\subsection{Reading between the lines}

We saw that some parts of the QALMRI weren't explicitly stated the by authors in the manuscript. However, all of the major parts were implied by what they wrote. This will be a common theme in reading the research literature. You will have to extract the implications when they are not explicitly stated. That means you have to read between the lines to understand the ideas and connections between them.

This research appeared to have no theory behind it. So, the results of the paper did not appear to test predictions of any theory. Instead, the results were testing the popular claim or stereotype that women talk more than men. The data show strong, reliable evidence that this claim simply isn't true, at least for the subjects they measured. But, do the data have implications beyond testing the stereotype? Do these data have theoretical implications? Yes, they do, but you have to read between the lines to see the connection.

The author's allude to this connection in the final paragraph, where they say "Further, to the extent that sex differences in daily word use are assumed to be biologically based, evolved adaptations [3], they should be detectable among university students as much as in more diverse samples". The clue is the reference to [3], which you can find in the article, and refers to a book called "The Female Brain". I have not read this book, but let's imagine that it contains a theory about evolved adaptations that are biologically based, which cause differences in how women and men talk. This theory would make a number of assumptions, and it would predict that women talk more than men. So, even though the authors do not describe this theory in detail to explain how it works, and why it might predict that women talk more than men, their data nevertheless has implications for that theory. Specifically, any data showing that women and men say about the same numbers of words per day, has implications for any theory that predicts otherwise. So, reading between the lines, we can see that findings suggest that theories suggesting differences between women and men must be making wrong assumptions. And, the results point us toward theories that suggest common psychological principles and processes driven male and female talking behavior. So, indeed there are many implications between the lines that we arrive at when we draw them out of the text by thinking about the ideas, and the connections between ideas and the data.

\section{The Hardest Section of the QALRMI: Alternatives}

In my opinion, the alternatives section is the hardest, and sometimes the most confusing part to extract from a journal article. Sometimes the author's make it really easy by writing very clearly and explicitly. For example, you may find that the authors explicitly define two hypotheses, which is very helpful. But, this is not always the case. So, when there is only one hypothesis, don't spend time looking in the manuscript for a second one (however, just because the authors didn't generate a second hypothesis, doesn't mean that you can't do that for yourself. There's always an alternative way of explaining things.)

The alternatives section appears to be simple when the claim that is being tested is an empirical claim. Empirical claims are questions about whether the observations or measurements (dependent variables) are different between the conditions of the experiment (independent variables). There are always two general possibilities. 

\begin{enumerate}
\item There is a difference between conditions.
\item There is no difference between conditions (like in the above paper, there were no differences between women and men in the number of words spoken per day)
\end{enumerate}

So, you might come across articles where the paper is asking whether one thing influences another. For example, does the color of a room (red vs. blue) change how fast you can read; or does drinking coffee vs. tea change how quickly you can learn piano.  There are many "does this do something to that" kind of questions. In these cases, the alternative section boils down to:

\marginnote{X is the manipulation (independent or manipulated variable), like a blue or red room. And, Y is the measurement (dependent variable), like how fast you read.}

\begin{itemize}
\item Empirical Hypothesis 1: X does change something in Y (This does do something to that)
\item Empirical Hypothesis 2: X does not change something in Y (This does nothing to that)
\end{itemize}

\subsection{Empirical vs. Theoretical Hypotheses}

Notice the new label "Empirical hypothesis". This label means that the hypothesis is directly concerned with whether the experiment will do something, that is whether the manipulation will actually change something that was measured, or will do nothing at all. In fact, we might make a new label that eliminates the word hypothesis, because these two statements aren't really hypotheses in any interesting sense, instead they just the two possible outcomes that can always occur in any experiment. So, let's use the labels "Empirical Outcome"

\begin{itemize}
\item Empirical Outcome 1: X does change something in Y (This does do something to that)
\item Empirical Outcome 2: X does not change something in Y (This does nothing to that)
\end{itemize}

You should be able to see by now that there are always two outcomes when running an experiment. It does something (produces some difference), or it does nothing (produces no differences). So, for every experiment, you could always write an "Alternative"section that simply says the alternatives were 1) the experiment will do something, or 2) the experiment will not do anything. But, because this is true of all experiments, it really does not tell us anything useful. 

Instead, what you should be looking for is the "THEORETICAL HYPOTHESIS" or "THEORETICAL HYPOTHESES". The theoretical hypotheses are where the alternative sections get really interesting, but at the same time can be very challenging to understand. Theoretical hypotheses can be challenging to understand because you have to understand what the theory is, how it works, and why it makes particular hypotheses. Most important are the connections between the theoretical hypotheses and the research findings. In particular, when a theoretical hypothesis fails to predict the research findings, the hypothesis must be somehow wrong! So, we can use the findings from experiments to rule out bad theories or explanations, which opens the door to new and improved theories that can explain the findings. 

 Returning the "read between the lines" theme, not all papers will be explicit about the difference between theoretical hypotheses and empirical hypotheses. For example, look back at our QALMRI summary for the Mehl et al. (2007) paper. What kind of Alternatives did we list? It should be clear that we listed two empirical hypotheses: Either women talk more than men, or they don't. Also, reading the paper, the authors' themselves were mainly concerned with the empirical question, and only alluded to the theoretical issues in passing at the end of the article. 

So, when you are writing QALMRI summaries be aware that some papers fail to state the theoretical issues. When this is the case it is totally appropriate to state the Alternatives in terms of the Empirical hypotheses, as perhaps the authors were not testing any theories, or have no theoretical explanations for their results. You could also attempt to figure out what the theoretical issues might be by reading between the lines. However, when the article discusses the theories that motivate the research, you must state the alternatives in terms of the Theoretical hypotheses, because in these cases they are the most important parts of the paper. 

Finally, let's consider how we could change the Alternatives section that we wrote for the Mehl et al. (2007) paper, so that rather than listing empirical hypotheses, we instead imagine what the theoretical hypotheses might be.

The original empirical hypotheses were:

\begin{itemize}
\item Empirical Hypothesis 1: Women are more talkative than men
\item Empirical Hypothesis 2: Women are not more talkative than men
\end{itemize}

Some possible theoretical hypotheses are:

\begin{enumerate}
\item Theoretical Hypothesis 1: Women and men have specific biological adaptations stemming from different evolutionary pressures that cause women to be more social than men. The implication is that women may exhibit more social behaviors, such as talking, than men.
\item Theoretical Hypothesis 2: The neuro-biological and cognitive processes involved in language production are virtually identical between the sexes. The implication is that talking behaviors are not fundamentally different between men and women.
\end{enumerate}

To make the distinction and relationship between theoretical and empirical hypotheses more clear, we can put the empirical hypotheses inside the theoretical ones.

\begin{enumerate}
\item Theoretical Hypothesis 1: Women and men have specific biological adaptations stemming from different evolutionary pressures that cause women to be more social than men. The implication is that women may exhibit more social behaviors, such as talking, than men.
	\begin{enumerate}
	\item Empirical Hypothesis 1: Women are more talkative than men
	\end{enumerate}
\item Theoretical Hypothesis 2: The neuro-biological and cognitive processes involved in language production are virtually identical between the sexes. The implication is that talking behaviors are not fundamentally different between men and women.
	\begin{enumerate}
	\item Empirical Hypothesis 2: Women are not more talkative than men
	\end{enumerate}
\end{enumerate}

Each theoretical hypothesis implies particular empirical outcomes. So, we collect the data about the empirical outcomes because they test the implications of the theoretical hypotheses.









\chapter{Labs 2 and 3: Paper Project 1}
\lhead{\allcaps{Labs 2 and 3: Paper Project 1}}
\openepigraph{The purpose of psychology is to give us a completely different idea of the things we know best.}{---Paul Valery}


\section{Overview}

In this project you will attempt to replicate a recent short report published in Psychological Science. A replication of the experiment has already been designed. To collect data, you will first participate as a subject in the experiment. Then, as a class you will be introduced to the published paper. You will read the paper and it in class. Then, the class will analyze the collected data to determine whether or not the major effects of interest have been replicated. Data will be collected using pen and paper methods, and analyzed by computer software. Each student will write a 5+ page, APA style report on the project.

\subsection{Things you will learn:}

\begin{itemize}
\item Reading and citing primary source material
\item Writing a brief APA style research report
\item Conducting and reporting T-tests
\end{itemize}

\subsection{Background readings:}

To be provided by Lab Instructor

\subsection{Grade}
\begin{itemize}
\item  10\% of final grade
\item Graded by your lab instructor. Lab instructor sets due date, and determines whether revised drafts are submittable.
\item If you submit completed versions of all paper assignments (1, 2, and 3), then you get to drop your lowest paper grade from paper 1 or 2. Specifically, you will receive your highest grade from paper 1 or 2 for both papers, thereby eliminating the lower grade. This allows room to learn and improve as you go. 
\end{itemize}


\section{Writing the paper}

There are many resources for help on writing an APA style research report in the lab manual, on the textbook, and the website. Check them out. As well, here is a rough roadmap for writing paper 1.

\subsection{APA formatting, Title and Abstract}

\begin{itemize}
\item Use APA formatting rules.

\item Create a suitable title for the paper

\item Write the abstract : No more than 250 words. The aim is to briefly describe the issue at hand, the experiment, and the results.
\end{itemize}

\subsection{Introduction (around 2 double-spaced pages)}

The goal of the introduction is to first put the research into a broader context, and then narrow the focus to describe the specific research aims.

\begin{enumerate}
\item A. Opening section: (starting broad)

\begin{itemize}
\item about 1 paragraph
\item Discuss a real-world example of the general phenomena under investigation by the paper
\item Tell the reader that the purpose of the current experiment is to conduct a replication of the work in question
\item  Establish the domain and big questions.
\end{itemize}

\item Middle section: Prior work

\begin{itemize}
\item Discuss some examples of previous research that are similar to the present research. You have an opportunity here to look this kind of research up on Google Scholar. One or two examples ought to be enough.
\item Explain the specific question that is being asked in this replication work.
\end{itemize}

\item Final section: (briefly explain the present aims, the experiment and what you expect to find)
\begin{itemize}
\item Explain the hypotheses (alternatives)
\item Explain the logic of how the hypotheses will be tested
\item Briefly explain what the participants will be doing in the task
\item Briefly give predictions for performance in each condition
\end{itemize}

\end{enumerate}


\subsection{Methods (about 1 page)}

The methods section should be a complete recipe that anyone could follow to replicate your experiment. There are lots of details that you can include, some of these are listed below. Be brief and concise

\begin{enumerate}
\item Participants
	\begin{itemize}
	\item how many people?
	\item where did they come from?
	\end{itemize}
\item Materials
	\begin{itemize}
	\item what were the stimuli?
	\item how were they organized?
	\end{itemize}

\item Design \& Procedure
	\begin{itemize}
	\item What was the design
	\item What were the independent variable(s)
	\item What was the dependent variable
	\item Within or between subjects?
	\item How were the stimuli for each trial chosen
	\item Describe the steps each participant took to complete the experiment
	\end{itemize}

\end{enumerate}

\subsection{Results}

The result section is used to report the patterns in the data, and the statistical support for those patterns. You will compute the results using SPSS in the lab computers. The lab manual can be consulted for help on running statistical tests, and for reporting results.

\begin{itemize}
\item Describe the statistical analysis
\item Tell the reader where they can see the data. E.g., the results of experiment 1 are presented in table 1, or in figure 1
\item Make a table or figure to display the data in your paper
\item Report the statistical test, and the pattern of the means.
\end{itemize}

\subsection{Discussion}

The discussion can be used to briefly restate verbally the pattern of the most important results, and then to relate the results to theory and ideas developed in the introduction

\begin{itemize}
\item Highlight the main findings from the experiment
\item Discuss how the data can be explained by the hypothesis. What inferences do you make about the hypotheses based on the research findings?
\item Broaden your discussion. Can the findings be explained by an alternative theory? What can be generalized to the real world? Are there important confounds that prevent us from interpreting our results?
\end{itemize}

\subsection{References and Figures or Tables}

\begin{itemize}
\item Include citations used in the paper using APA style format
\item Include a figure or table to show the results
\end{itemize}


\section{Data-analysis tips}\label{lab-1-data-analysis-tips}

In lab one you will be collecting measurements on several dependent
variables, in each of two manipulated conditions (the independent
variable). For each dependent variable you will want to determine
whether the manipulation had an effect. That is, did the independent
variable cause a change in the dependent variable. We know that
differences can sometimes be observed by chance alone, so we want to
conduct an inferential statistical test to determine the probability
that our observed difference could have been produced by chance alone.
To do this we will be conducting several t-tests. This is a short primer
on the process. You can conduct t-tests in the software of your choice,
or by hand using a calculator (or in excel). Here, we will use the free
and open-source statistical package called R, to illustrate the process.

Let's imagine we have two groups of 10 subjects each. Group A receives
condition 1 of the independent variable, and Group B recieves condition
2 of the independent variable. We then measure some behavior for all of
the subject in all of the groups. To make this more concrete, let's say
10 subjects drink coffee, and the the 10 subjects drink tea. Then we
present all of the subjects with a piece of art and ask them rate how
beautiful they think it is on a scale from 1 to 7.

When we collect all the data we should have 20 total ratings, one for
each subject in each group.

For example, if you put the data in a table it might look something like
the following. Note, the grey text box shows the R code used to simulate
the data. For, each group, we sample 10 numbers from a normal
distribution with a mean of 4, and a standard deviation of .5. Then we
put the numbers in a table.

\begin{Shaded}
\begin{Highlighting}[]
\NormalTok{coffee<-}\KeywordTok{round}\NormalTok{(}\KeywordTok{rnorm}\NormalTok{(}\DecValTok{10}\NormalTok{,}\DecValTok{4}\NormalTok{,.}\DecValTok{5}\NormalTok{))}
\NormalTok{tea<-}\KeywordTok{round}\NormalTok{(}\KeywordTok{rnorm}\NormalTok{(}\DecValTok{10}\NormalTok{,}\DecValTok{4}\NormalTok{,.}\DecValTok{5}\NormalTok{))}
\NormalTok{all_data<-}\KeywordTok{data.frame}\NormalTok{(coffee,tea)}
\KeywordTok{kable}\NormalTok{(all_data,}\DataTypeTok{format=}\StringTok{"latex"}\NormalTok{)}
\end{Highlighting}
\end{Shaded}

\begin{tabular}{r|r}
\hline
coffee & tea\\
\hline
4 & 5\\
\hline
3 & 4\\
\hline
3 & 4\\
\hline
3 & 3\\
\hline
4 & 3\\
\hline
4 & 5\\
\hline
4 & 4\\
\hline
4 & 4\\
\hline
4 & 4\\
\hline
3 & 4\\
\hline
\end{tabular}

We can do some quick descriptive statistics, for example, we might want
to know the means of the beauty ratings for the coffee and tea groups.

\begin{Shaded}
\begin{Highlighting}[]
\KeywordTok{mean}\NormalTok{(coffee)}
\end{Highlighting}
\end{Shaded}

\begin{verbatim}
## [1] 3.6
\end{verbatim}

\begin{Shaded}
\begin{Highlighting}[]
\KeywordTok{mean}\NormalTok{(tea)}
\end{Highlighting}
\end{Shaded}

\begin{verbatim}
## [1] 4
\end{verbatim}

The means aren't very different, and of course we should expect they
should be similar. After all, we sampled these means from the exact same
distribution. So, we should expect that on average, the means should
both be close to 4. However, they won't necessarilly be exactly 4,
because of variability introduced by random sampling.

\section{The t-test}\label{the-t-test}

What we want to do next is conduct an independent samples t-test. We
want to determine whether any possible difference between the coffee and
tea groups could have been produced by chance alone. We can conduct a
t-test in R very easily using the t.test function.

\begin{Shaded}
\begin{Highlighting}[]
\KeywordTok{t.test}\NormalTok{(coffee,tea,}\DataTypeTok{var.equal=}\OtherTok{TRUE}\NormalTok{)}
\end{Highlighting}
\end{Shaded}

\begin{verbatim}
## 
##  Two Sample t-test
## 
## data:  coffee and tea
## t = -1.5, df = 18, p-value = 0.151
## alternative hypothesis: true difference in means is not equal to 0
## 95 percent confidence interval:
##  -0.9602459  0.1602459
## sample estimates:
## mean of x mean of y 
##       3.6       4.0
\end{verbatim}

R gives us back the t values, the degrees of freedom (df), and the
associated p-value. The p-value tells us the likelihood that our
difference, or a difference greater than the one we observed could have
been produced by chance.

\section{One more time}\label{one-more-time}

Let's try this whole process again, but this time we will simulate data
with an actual difference between the groups. For example, let's say we
want to simulate the idea that drinking coffee makes people think the
art is less beautiful by at least 2 points, and then reconduct the
t-test with the new simulated data. We will sample numbers from a normal
distribution with mean 3 for the coffee group, and mean 5 for the tea
group (for an average expected difference of 2).

\begin{Shaded}
\begin{Highlighting}[]
\NormalTok{coffee<-}\KeywordTok{round}\NormalTok{(}\KeywordTok{rnorm}\NormalTok{(}\DecValTok{10}\NormalTok{,}\DecValTok{3}\NormalTok{,.}\DecValTok{5}\NormalTok{))}
\NormalTok{tea<-}\KeywordTok{round}\NormalTok{(}\KeywordTok{rnorm}\NormalTok{(}\DecValTok{10}\NormalTok{,}\DecValTok{5}\NormalTok{,.}\DecValTok{5}\NormalTok{))}
\NormalTok{all_data<-}\KeywordTok{data.frame}\NormalTok{(coffee,tea)}
\KeywordTok{kable}\NormalTok{(all_data,}\DataTypeTok{format=}\StringTok{"latex"}\NormalTok{)}
\end{Highlighting}
\end{Shaded}

\begin{tabular}{r|r}
\hline
coffee & tea\\
\hline
3 & 5\\
\hline
3 & 5\\
\hline
3 & 4\\
\hline
3 & 6\\
\hline
2 & 5\\
\hline
4 & 5\\
\hline
3 & 6\\
\hline
3 & 5\\
\hline
3 & 5\\
\hline
3 & 5\\
\hline
\end{tabular}

\begin{Shaded}
\begin{Highlighting}[]
\KeywordTok{mean}\NormalTok{(coffee)}
\end{Highlighting}
\end{Shaded}

\begin{verbatim}
## [1] 3
\end{verbatim}

\begin{Shaded}
\begin{Highlighting}[]
\KeywordTok{mean}\NormalTok{(tea)}
\end{Highlighting}
\end{Shaded}

\begin{verbatim}
## [1] 5.1
\end{verbatim}

\begin{Shaded}
\begin{Highlighting}[]
\KeywordTok{t.test}\NormalTok{(coffee,tea,}\DataTypeTok{var.equal=}\OtherTok{TRUE}\NormalTok{)}
\end{Highlighting}
\end{Shaded}

\begin{verbatim}
## 
##  Two Sample t-test
## 
## data:  coffee and tea
## t = -9, df = 18, p-value = 4.404e-08
## alternative hypothesis: true difference in means is not equal to 0
## 95 percent confidence interval:
##  -2.590215 -1.609785
## sample estimates:
## mean of x mean of y 
##       3.0       5.1
\end{verbatim}

\section{Writing up the results of a
t-test}\label{writing-up-the-results-of-a-t-test}

We've now conducted two different t-tests, and received different
results on each them. You will likely find different results for all of
the t-tests that you conduct for the lab experiment. However, you will
use the basic sentence structure to report all of the results. When you
report the results of your experiment along with statistical tests there
are two important features to include, the pattern of the results, and
the inferential statistic. In this situation, we would simply report the
means and the t-test information. Here are is an with made-up numbers.

The coffee group gave a lower mean beauty rating (M = 3.4) than the tea
group (M = 5.6), and the difference was significant, t (18) = 5.4, p
\textless{} .001.

So, just in one sentence we tell the reader what the means were in both
conditions, as whether the result was significant. APA style recommends
reporting exact p-values when they are greater than .001 (for example p
= .047). If the p-value is less than .001, then you just need to report
p \textless{} .001.







\chapter{Lab 4: Memory Mini Project}
\lhead{\allcaps{Lab 4: Memory Mini Project}}
\openepigraph{Memory... is the diary that we all carry about with us.}{---Oscar Wilde}

\section{Lab 4: MiniProject 1 Nairne, Pandeirada, \& Thompson, (2008)}

In the first mini-project, you will read, summarize and discuss the paper by \citeauthor{nairne_adaptive_2008} (2008)\cite{nairne_adaptive_2008}. Then, you will attempt to replicate their results in the lab, by conducting an experiment and analyzing the data.

\subsection{What's in store}
\begin{enumerate}
\item Students read paper and write QALMRI (15-20)
\item Group discussion about paper (15-20)
\item	Students attempt to replicate the major findings in the paper

a.	Break into four-five groups, each group assigned an encoding condition (Survival, Pleasantness, Imagery, Self-reference, Intentional learning)

b.	Each group picks their own 30 words. Can follow same procedure as in paper by choosing 30 words from Overschelde, Rawson, \& Dunlosky (2004) \cite{van_overschelde_category_2004}.

c.	Groups try to run at least 10 participants in their condition, recording proportion of correctly recalled words

d.	Groups enter their collected data into the master spreadsheet, which is given back to groups upon data completion

\item Discussion of how to analyze the data
\item Group attempt to analyze the data using t-tests and one-way ANOVA to determine if the survival framing produced better recall than the other conditions.
\end{enumerate}




\section{Data-analysis tips}\label{lab-4-data-analysis-tips}

In this mini-lab, you will be determining whether different kinds of
manipulations improve your ability to recall lists of words. Each
condition different subjects will read a list a words, and then recall
as many as they can on a sheet of paper. Then, you will measure recall
for each subject by counting the number of correctly recalled words. You
will try to get at least 10 subjects in each condition, the more the
merrier. The empirical question will be whether memory recall is differs
across the conditions. If they do differ, then some of those conditions
help you remember words better, and some conditions may hurt your
ability to remember words. With these kinds of experiments, we can learn
about the factors that influence our memory, as well as how cognitive
processes allow us to encode and retrieve memories.

The new tool you be using here is a one-way ANOVA, which you may
remember from your statistics classes. In particular, we will be
performing an independent samples ANOVA, because we are conducting a
between-subjects experiment, where different subjects contribute data in
each of the conditions. More specifically, we will be conducting an
omni-bus test. The omni-bus test is a catch-all test, that let's us
enter the data from as many conditions as we want, and ask the general
question: are any of the conditions different from one another. This is
different from a t-test, which only let's us make a comparison between
two conditions. In fact, the t-test and ANOVA are related, but this is a
topic that we will leave for now. Except to say, that if you ran a
t-test comparing two groups, or an ANOVA with only two groups, they
would return the same p-values, because they are testing the same
underlying question\ldots{}which is the likelihood the observed occurs
by chance alone.

One drawback about the ANOVA, is that it only tells us if there is some
difference, any difference, among the conditions we are comparing. But,
it does not tell us which specific conditions are different.
Fortunately, after we conduct the ANOVA, we can run follow-up t-tests
comparing any groups we want. We can also conduct more complex
comparisons using linear contrasts, but that is a topic for another day
as well.

We will again use to R to simulate a memory recall experiment with
multiple conditions. Then, we will create fake data, plot the data, then
analyze the data for differences using an ANOVA and follow-up tests.

\subsection{Simulating the data}\label{simulating-the-data}

Consider an experiment with five different memorizing conditions. This
will be a between-subjects experiment with 10 simulated subjects in each
condition. We will have condition A, B, C, D, and E. To simulate data
for each subject we need to make some assumptions. Let's that out of 30
words most people remember about 15 of them, but there is variation, so
some people do better and some people do worse. We can model this by
sampling numbers randomly from a distribution of our choice. For
convenience, we will use the normal distribution, which looks like
bell-curve (should be familiar from statistics class). Let's imagine
that condition A and B help memory more than C and D, and that memory is
worse in condition E. Take a look at the code and output for simulating
this kind of data.

\begin{Shaded}
\begin{Highlighting}[]
\NormalTok{A<-}\KeywordTok{round}\NormalTok{(}\KeywordTok{rnorm}\NormalTok{(}\DecValTok{10}\NormalTok{,}\DecValTok{20}\NormalTok{,}\DecValTok{2}\NormalTok{))}
\NormalTok{B<-}\KeywordTok{round}\NormalTok{(}\KeywordTok{rnorm}\NormalTok{(}\DecValTok{10}\NormalTok{,}\DecValTok{20}\NormalTok{,}\DecValTok{2}\NormalTok{))}
\NormalTok{C<-}\KeywordTok{round}\NormalTok{(}\KeywordTok{rnorm}\NormalTok{(}\DecValTok{10}\NormalTok{,}\DecValTok{15}\NormalTok{,}\DecValTok{2}\NormalTok{))}
\NormalTok{D<-}\KeywordTok{round}\NormalTok{(}\KeywordTok{rnorm}\NormalTok{(}\DecValTok{10}\NormalTok{,}\DecValTok{15}\NormalTok{,}\DecValTok{2}\NormalTok{))}
\NormalTok{E<-}\KeywordTok{round}\NormalTok{(}\KeywordTok{rnorm}\NormalTok{(}\DecValTok{10}\NormalTok{,}\DecValTok{10}\NormalTok{,}\DecValTok{2}\NormalTok{))}
\NormalTok{all_data<-}\KeywordTok{data.frame}\NormalTok{(A,B,C,D,E)}
\KeywordTok{kable}\NormalTok{(all_data,}\DataTypeTok{format=}\StringTok{"latex"}\NormalTok{)}
\end{Highlighting}
\end{Shaded}

\begin{tabular}{r|r|r|r|r}
\hline
A & B & C & D & E\\
\hline
23 & 19 & 17 & 18 & 13\\
\hline
18 & 21 & 17 & 15 & 11\\
\hline
20 & 24 & 16 & 11 & 10\\
\hline
22 & 20 & 15 & 11 & 12\\
\hline
22 & 21 & 12 & 13 & 12\\
\hline
17 & 15 & 15 & 18 & 10\\
\hline
20 & 19 & 14 & 13 & 9\\
\hline
19 & 23 & 17 & 17 & 12\\
\hline
20 & 23 & 13 & 17 & 6\\
\hline
21 & 22 & 19 & 13 & 12\\
\hline
\end{tabular}

We have produced a table with fake data for 10 subjects in each
condition. The numbers all represent the number of correctly recalled
words for each simulated subject. For groups A and B we sample 10
numbers, from a distribution with mean 20, and standard deviation 2.
This is a higher mean than groups C and D (mean = 15). The lowest mean
was for Group E (mean = 10). So, on average, groups A and B should have
higher scores than C and D, which should be higher than E.

Ok, so what happened in our simulated experiment. We can see the numbers
in the table, but it would be nice to summarize them so we can more
easily look at differences. After all, it's hard to make sense of a
bunch of raw data in a table.

One way to summarize the data is to compute the group means for each
condition. This averages over the subjects, and gives us only 5 means to
look at, so it is easier to see the differences. We can ``easily'' do
this in R in a couple different ways. However, R often likes the data in
a particular format, in this case long-data format. So, we will first
convert to that format, and see what it looks like.

\begin{Shaded}
\begin{Highlighting}[]
\NormalTok{long_data<-}\KeywordTok{data.frame}\NormalTok{(}\DataTypeTok{Conditions=}\KeywordTok{rep}\NormalTok{(}\KeywordTok{c}\NormalTok{(}\StringTok{"A"}\NormalTok{,}\StringTok{"B"}\NormalTok{,}\StringTok{"C"}\NormalTok{,}\StringTok{"D"}\NormalTok{,}\StringTok{"E"}\NormalTok{),}\DataTypeTok{each=}\DecValTok{10}\NormalTok{),}
                      \DataTypeTok{Recall=}\KeywordTok{c}\NormalTok{(A,B,C,D,E))}
\KeywordTok{kable}\NormalTok{(long_data[}\DecValTok{1}\NormalTok{:}\DecValTok{25}\NormalTok{,],}\DataTypeTok{format=}\StringTok{"latex"}\NormalTok{)}
\end{Highlighting}
\end{Shaded}

\begin{tabular}{l|r}
\hline
Conditions & Recall\\
\hline
A & 23\\
\hline
A & 18\\
\hline
A & 20\\
\hline
A & 22\\
\hline
A & 22\\
\hline
A & 17\\
\hline
A & 20\\
\hline
A & 19\\
\hline
A & 20\\
\hline
A & 21\\
\hline
B & 19\\
\hline
B & 21\\
\hline
B & 24\\
\hline
B & 20\\
\hline
B & 21\\
\hline
B & 15\\
\hline
B & 19\\
\hline
B & 23\\
\hline
B & 23\\
\hline
B & 22\\
\hline
C & 17\\
\hline
C & 17\\
\hline
C & 16\\
\hline
C & 15\\
\hline
C & 12\\
\hline
\end{tabular}

I've only printed the first 25 lines, but the dataframe contains all of
the data for conditions, C, D, and E as well. You can see why they call
it long format. It's because each data point gets it's own row in the
table.

\subsection{Looking at the means}\label{looking-at-the-means}

Now that the data is in long format we can easily make a table of the
condition means

\begin{Shaded}
\begin{Highlighting}[]
\NormalTok{condition_means<-}\KeywordTok{aggregate}\NormalTok{(Recall~Conditions,long_data,mean)}
\KeywordTok{kable}\NormalTok{(condition_means,}\DataTypeTok{format=}\StringTok{"latex"}\NormalTok{)}
\end{Highlighting}
\end{Shaded}

\begin{tabular}{l|r}
\hline
Conditions & Recall\\
\hline
A & 20.2\\
\hline
B & 20.7\\
\hline
C & 15.5\\
\hline
D & 14.6\\
\hline
E & 10.7\\
\hline
\end{tabular}

We can now see the group means, but we can't see any measure of how
variable the data are in each condition. We might, for example, also
want to compute the standard deviation as well as the mean, and put them
both in the table. We could run the same code from above and substite sd
for mean, which would give us a table of standard deviations. However,
we will use a more advanced function from the plyr package, called
ddply. ddply let's you compute multiple statistics and put them all in a
single table. The syntax is a bit different, but it doesn't take long to
get used to it.

\begin{Shaded}
\begin{Highlighting}[]
\KeywordTok{library}\NormalTok{(plyr)}
\NormalTok{condition_means<-}\KeywordTok{ddply}\NormalTok{(long_data,.(Conditions),summarise,}
                       \DataTypeTok{MeanRecall=}\KeywordTok{mean}\NormalTok{(Recall),}
                       \DataTypeTok{StdDeviation=}\KeywordTok{sd}\NormalTok{(Recall))}
\KeywordTok{kable}\NormalTok{(condition_means,}\DataTypeTok{format=}\StringTok{"latex"}\NormalTok{)}
\end{Highlighting}
\end{Shaded}

\begin{tabular}{l|r|r}
\hline
Conditions & MeanRecall & StdDeviation\\
\hline
A & 20.2 & 1.873796\\
\hline
B & 20.7 & 2.626785\\
\hline
C & 15.5 & 2.121320\\
\hline
D & 14.6 & 2.756810\\
\hline
E & 10.7 & 2.057507\\
\hline
\end{tabular}

\subsection{Plotting the data}\label{plotting-the-data}

It's often very desirable to plot the data in a graph, rather than just
present the means in a table. People find it easier to look at graphs,
because the differences in the data just pop-out much easier than
looking at numbers in a table. R has a fantastic graphing package called
ggplot2. ggplot2 is a whole philosophy for visual design and
data-presentation, and it can be daunting at first. But, it's complexity
makes it very powerful, and when you get the hang of it you can very
quickly make all sorts of beautiful graphs to present data. Here is some
code to make ggplot create a bar graph to plot the means, along with
error bars. In this case the error bars with represent standard errors
of the mean, rather than standard deviations. R does not have a built in
function for the standard error of the mean, so we have to write it
ourselves.

\begin{Shaded}
\begin{Highlighting}[]
\KeywordTok{library}\NormalTok{(ggplot2)}
\NormalTok{sde<-function(x)\{}\KeywordTok{sd}\NormalTok{(x)/}\KeywordTok{length}\NormalTok{(x)\}}
\NormalTok{plot_means<-}\KeywordTok{ddply}\NormalTok{(long_data,.(Conditions),summarise,}
                       \DataTypeTok{MeanRecall=}\KeywordTok{mean}\NormalTok{(Recall),}
                       \DataTypeTok{SE=}\KeywordTok{sde}\NormalTok{(Recall))}

\NormalTok{limits <-}\StringTok{ }\KeywordTok{aes}\NormalTok{(}\DataTypeTok{ymax =} \NormalTok{MeanRecall +}\StringTok{ }\NormalTok{SE, }\DataTypeTok{ymin =} \NormalTok{MeanRecall -}\StringTok{ }\NormalTok{SE)}

\KeywordTok{ggplot}\NormalTok{(plot_means,}\KeywordTok{aes}\NormalTok{(}\DataTypeTok{x=}\NormalTok{Conditions, }\DataTypeTok{y=}\NormalTok{MeanRecall))+}
\StringTok{  }\KeywordTok{geom_bar}\NormalTok{(}\DataTypeTok{position=}\StringTok{"dodge"}\NormalTok{,}\DataTypeTok{stat=}\StringTok{"identity"}\NormalTok{)+}
\StringTok{  }\KeywordTok{geom_errorbar}\NormalTok{(limits, }\DataTypeTok{width=}\NormalTok{.}\DecValTok{1}\NormalTok{)+}
\StringTok{  }\KeywordTok{theme_classic}\NormalTok{(}\DataTypeTok{base_size=}\DecValTok{12}\NormalTok{)+}
\StringTok{  }\KeywordTok{ylab}\NormalTok{(}\StringTok{"Mean Correctly }\CharTok{\textbackslash{}n}\StringTok{ Recalled Words"}\NormalTok{)+}
\StringTok{  }\KeywordTok{xlab}\NormalTok{(}\StringTok{"Condition"}\NormalTok{)}
\end{Highlighting}
\end{Shaded}

\includegraphics{Lab4_files/figure-latex/unnamed-chunk-5-1}

Now it is easy to the differences between conditions. Just as we had
hoped, Groups A and B appear to have recalled more words than Groups C
and D, which remembered more words than group E.

\subsection{Conducting the ANOVA}\label{conducting-the-anova}

Although the graph and the tabe show some clear differences in the
means, we still want to find out the probability that this kind of
finding occurs by chance alone. We can be confident in the differences
when we know that they do not occur very often by chance alone. The
first step is conduct a one-way ANOVA. This is very easy in R.

\begin{Shaded}
\begin{Highlighting}[]
\NormalTok{aov.out<-}\KeywordTok{aov}\NormalTok{(Recall~Conditions,long_data)}
\end{Highlighting}
\end{Shaded}

We're done! It's only one line of code. However, we need a couple more
to see the results.

\begin{Shaded}
\begin{Highlighting}[]
\NormalTok{aov_summary<-}\KeywordTok{summary}\NormalTok{(aov.out)}
\KeywordTok{kable}\NormalTok{(}\KeywordTok{xtable}\NormalTok{(aov_summary),}\DataTypeTok{format=}\StringTok{"latex"}\NormalTok{)}
\end{Highlighting}
\end{Shaded}

\begin{tabular}{l|r|r|r|r|r}
\hline
  & Df & Sum Sq & Mean Sq & F value & Pr(>F)\\
\hline
Conditions & 4 & 694.52 & 173.630000 & 32.46095 & 0\\
\hline
Residuals & 45 & 240.70 & 5.348889 & NA & NA\\
\hline
\end{tabular}

The ANOVA table gives us a bunch of information. We will go into much
greater detail about the meaning of each number in the table, but also
assume for now that you are somewhat familiar with these ideas because
you have already taken statistics, right?

We are mainly interested in the p-value, which tells how often results
like the ones we found can occur by chance. But, when we report the
results of our ANOVA, we also provide additional information about the
F-value, the degrees of freedom values, and the mean squared error term.
The reason is that if you know these numbers, you can actually
reconstruct all of the other numbers. The results of our ANOVA are
significant. You could report this in a sentence like the following.

The main effect of condition was significant, F(4, 45) = 32.46, MSE =
5.35, p \textless{} .001.

\subsection{Comparisons between
conditions}\label{comparisons-between-conditions}

The p-value from above is much smaller than .05, which shows the
difference between conditions in the data does not occur very often by
chance alone. However, because we conducted an omni-bus test, we only
know that there is some difference between conditions, but we do not
know which specific conditions are different from one another.

So, we have to conduct additional tests between specific conditions.
There are multiple strategies for conducting these tests. For now, we
will simply run t-tests between comparisons of interest.

Remember, our data simulated the pattern that memory recall would be
better for groups A and B, which would be better than groups C and D,
which would better than group E. In other words A=B \textgreater{} C=D
\textgreater{} E.

We can confirm this pattern by conducting tests to see if it holds up.
For example, how would we test the pattern A=B \textgreater{} C=D
\textgreater{} E, all of the following comparisons need to be true,

\begin{itemize}
\item
  A = B
\item
  A \textgreater{} C
\item
  A \textgreater{} D
\item
  B \textgreater{} C
\item
  B \textgreater{} D
\item
  C = D
\end{itemize}

and, all of the conditions should be greater than E

\begin{itemize}
\item
  A \textgreater{} E
\item
  B \textgreater{} E
\item
  C \textgreater{} E
\item
  D \textgreater{} E
\end{itemize}

Let's conduct a few of these tests, and then report the findings.

\begin{Shaded}
\begin{Highlighting}[]
\KeywordTok{library}\NormalTok{(broom)}
\CommentTok{#conduct t-tests}
\NormalTok{ab<-}\KeywordTok{tidy}\NormalTok{(}\KeywordTok{t.test}\NormalTok{(A,B,}\DataTypeTok{var.equal =} \OtherTok{TRUE}\NormalTok{))}
\NormalTok{ac<-}\KeywordTok{tidy}\NormalTok{(}\KeywordTok{t.test}\NormalTok{(A,C,}\DataTypeTok{var.equal =} \OtherTok{TRUE}\NormalTok{))}
\NormalTok{cd<-}\KeywordTok{tidy}\NormalTok{(}\KeywordTok{t.test}\NormalTok{(C,D,}\DataTypeTok{var.equal =} \OtherTok{TRUE}\NormalTok{))}
\NormalTok{de<-}\KeywordTok{tidy}\NormalTok{(}\KeywordTok{t.test}\NormalTok{(D,E,}\DataTypeTok{var.equal =} \OtherTok{TRUE}\NormalTok{))}

\CommentTok{#put the results in a table}
\NormalTok{alltests<-}\KeywordTok{rbind}\NormalTok{(ab,ac,cd,de)}
\NormalTok{alltests<-}\KeywordTok{cbind}\NormalTok{(alltests,}\DataTypeTok{Comparison=}\KeywordTok{c}\NormalTok{(}\StringTok{"AB"}\NormalTok{,}\StringTok{"AC"}\NormalTok{,}\StringTok{"CD"}\NormalTok{,}\StringTok{"DE"}\NormalTok{))}
\NormalTok{finaltable <-}\StringTok{ }\KeywordTok{subset}\NormalTok{(alltests, }\DataTypeTok{select =} \KeywordTok{c}\NormalTok{(Comparison,estimate1,estimate2,statistic,p.value,parameter))}
\KeywordTok{kable}\NormalTok{(finaltable,}\DataTypeTok{format=}\StringTok{"latex"}\NormalTok{)}
\end{Highlighting}
\end{Shaded}

\begin{tabular}{l|r|r|r|r|r}
\hline
Comparison & estimate1 & estimate2 & statistic & p.value & parameter\\
\hline
AB & 20.2 & 20.7 & -0.4900286 & 0.6300329 & 18\\
\hline
AC & 20.2 & 15.5 & 5.2511144 & 0.0000541 & 18\\
\hline
CD & 15.5 & 14.6 & 0.8181818 & 0.4239528 & 18\\
\hline
DE & 14.6 & 10.7 & 3.5851808 & 0.0021158 & 18\\
\hline
\end{tabular}

\subsection{Writing it all up}\label{writing-it-all-up}

The following is an example results section for our hypothetical
experiment. This could serve as a model for your own results section.

The number of correctly recalled words for each subject in each
condition were submitted to a one-way ANOVA, with memorization condition
(A, B, C, D, and E) as the sole between-subjects factor. Mean recall
scores in each condition are displayed in Figure 1.

The main effect of memorization condition was significant, F(4, 45) =
32.46, MSE = 5.35, p \textless{} .001. Figure 1 shows that Groups A and
B had higher recall scores than Groups C and D, which had higher recall
scores than Group E. This pattern was confirmed across four independent
sample t-tests. Group A (M = 20.2) and Group B (M = 20.7) were not
significantly different t(18) = -0.49, p =0.63. Group A recalled
significantly more words than Group C (M = 15.5), t(18) = 5.25, p =0.
Group C and Group D (M = 14.6) were not significantly different t(18) =
0.82, p =0.424. Finally, Group D recalled significantly more words than
Group E (M = 10.7), t(18) = 3.59, p =0.002.











\chapter{Lab 5: Task-Switching Mini Project}
\lhead{\allcaps{Lab 5: Task-Switching Mini Project}}
\openepigraph{The secret to multitasking is that it isn't actually multitasking. It's just extreme focus and organization.
}{---Joss Whedon}


In the second mini-project, you will read, summarize and discuss the paper by \citeauthor{stoet_are_2013} (2013)\cite{stoet_are_2013}. Then, you will attempt to replicate their results in the lab, by conducting an experiment and analyzing the data.

\section{What's in store}

\begin{enumerate}
\item Students read paper and write QALMRI (15-20)
\item Group discussion about paper (15-20)
\item Students download the task-switching program available from the website and individually complete the task. Individual students then enter their data in the master spreadsheet
\item Discussion of how to analyze the data. Major analysis goals are:

a.	Was there a mixing cost? Compare pure lists to mixed lists

b.	Was there a switching cost? Compare switch vs. repeat trials in pure lists

c.	Did these effects depend on gender? Did the women have a smaller mixing cost than the men? Did the women have a smaller switching cost than the men?

\item Students break into groups to analyze the data.

\end{enumerate}

\section{Critical Concepts}

\subsection{Multiple Dependent and Independent Variables}

In lab 5 we begin looking at more complex designs that involve multiple dependent or and independent variables. For example, we can look at the dependent variable of reaction time or error rate, which are our two difference measured variables. There are also three independent variables, including gender (male vs female), block (pure list vs. mixed list), and trial sequence (repeat trial vs. switch trial). 

\subsection{Main effects and interactions}
Throughout this course we will be talking a lot about main effects and interactions. Feel free to jump ahead in the textbook and learn more about them. 

Main effects refer to the influence of single independent variables. And, each independent variable always has one main effect. So, we could look at the main effect of gender, block, or trial sequence. The main effect is like conducting a one-way ANOVA on the variable of interest, collapsing over the other variables. For example, a main effect of trial sequence usually shows slower mean reaction times in the switch condition compared to the repeat condition. This is called the task-switching cost, or also the task-switching effect.

Interactions occur when the effect of one independent variable depends on the levels of another independent variable. For example, if women actually multi-task better than men, then we should expect an interaction between gender and task-switching. Specifically, women should be better at task-switching than men. This would mean that women should have smaller task-switching costs (meaning they are unaffected by switching) compared to men (who should have bigger task-switching costs if they are more affected by switching).

If women have smaller task switching costs than men, then the effect of task-switching independent variable (repeat vs. switch) will depend on the level of the gender variable (women vs. men), which is the definition of an interaction.

\subsection{Manipulating an Effect}

Things get complicated when we add more dependent and independent variables, and it can be easy to lose track of what is going on in an experiment.

Many experiments are conducted for the purpose of better understanding some phenomena. For example, researchers observe phenomena like the task-switching effect, they create theories to explain the phenomena, then they test their theories by running more experiments. 

A primary question is often to identify factors that manipulate, influence, or somehow change the phenomena of interest. We know that switching between tasks can hurt performance, and we measure this cost by calculating a switch cost, which is the difference in performance between switching conditions and repeating conditions. What kinds of factors would make this cost ? Perhaps extended practice, time of day, gender, personality, motivation, drug interventions, details of the task, and many other factors could make the normal cost get smaller. Many of these factors might also make the effect get larger. And, knowledge of what makes switching costs smaller or larger can be used to test theories, which need to explain how and why those factors cause switching costs to change. 

The Stoet et al. (2013) took this approach to determine if switch costs are smaller for men than women. Notice we have been talking about switch costs, which is the phenonemon and effect of interest. We have not been talking too much about the dependent variable of reaction times that we use to compute the switch costs. Remember that switch costs are computed as RT switch - RT repeat.

\subsection{Reducing a two-factor design to a single-factor design using difference scores}

We can use the term switch cost as a convenient term to describe the effect of the switching variable on reaction times. We can also think of the switch cost as the dependent variable of interest, rather than the base reaction time scores. Thinking of the switch cost this way can change how you conceptualize the design.

For example, how many independent variables are in a design testing whether women have smaller switch costs than men? The answer could be two or one, depending on how you think of the design.

If we go with two, then the design involves the independent variables gender (women vs. men) and trial sequence (switch vs. repeat), and the dependent variable of reaction time. This design would have two main effects (effect of gender, and effect of switching), and one interaction (gender x switching). We would be mainly interested in the interaction. For example, women may have a smaller difference in mean reaction time between the switch and repeat conditions than men. 

We can reduce the design to a simple single-factor design. To do so, we create a new dependent variable called switch costs. For example, for each subject we compute the mean reaction in the switch and repeat condition. Then we  calculate the switch costs (difference between switch and repeat) for each subject. Now, each subject has one switch cost, rather than two reaction times. Now, the design would have gender as the single independent variable, and switch costs as the dependent variable. To determine whether women have smaller costs than men would you could run a t-test comparing the switch-costs between conditions.




\section{Data-analysis tips}\label{lab-5-data-analysis-tips}

To give you an idea about the analyses you will be performing, we will
again create simulated data to mimic aspects of the experiment, and then
go through the steps of performing the analysis.

We will simulate data for the switching cost. Specifically, we will
imagine that women have a smaller switching cost than men. The code
below generates sample data for 10 women and 10 men, who each have mean
reactions in the repeat and switch conditions. For women, the mean
reaction times were 580 ms for repeat and 600 ms for switch sequences,
for a total expected switch cost of 20 ms. For men, the mean reaction
times were 580 ms for repeat and 650 ms for switch sequences, for a
total expected switch cost of 70 ms.

\subsection{Simulating the data}\label{simulating-the-data}

\begin{Shaded}
\begin{Highlighting}[]
\NormalTok{women_switch <-}\KeywordTok{round}\NormalTok{(}\KeywordTok{rnorm}\NormalTok{(}\DecValTok{10}\NormalTok{,}\DecValTok{600}\NormalTok{,}\DecValTok{20}\NormalTok{))}
\NormalTok{women_repeat <-}\KeywordTok{round}\NormalTok{(}\KeywordTok{rnorm}\NormalTok{(}\DecValTok{10}\NormalTok{,}\DecValTok{580}\NormalTok{,}\DecValTok{20}\NormalTok{))}
\NormalTok{men_switch <-}\KeywordTok{round}\NormalTok{(}\KeywordTok{rnorm}\NormalTok{(}\DecValTok{10}\NormalTok{,}\DecValTok{650}\NormalTok{,}\DecValTok{20}\NormalTok{))}
\NormalTok{men_repeat <-}\KeywordTok{round}\NormalTok{(}\KeywordTok{rnorm}\NormalTok{(}\DecValTok{10}\NormalTok{,}\DecValTok{580}\NormalTok{,}\DecValTok{20}\NormalTok{))}
\NormalTok{all_data<-}\KeywordTok{data.frame}\NormalTok{(}\DataTypeTok{Subject=}\KeywordTok{c}\NormalTok{(}\KeywordTok{rep}\NormalTok{(}\KeywordTok{seq}\NormalTok{(}\DecValTok{1}\NormalTok{,}\DecValTok{10}\NormalTok{,}\DecValTok{1}\NormalTok{),}\DecValTok{2}\NormalTok{),}
                               \KeywordTok{rep}\NormalTok{(}\KeywordTok{seq}\NormalTok{(}\DecValTok{11}\NormalTok{,}\DecValTok{20}\NormalTok{,}\DecValTok{1}\NormalTok{),}\DecValTok{2}\NormalTok{)),}
                     \DataTypeTok{Gender=}\KeywordTok{rep}\NormalTok{(}\KeywordTok{c}\NormalTok{(}\StringTok{"Female"}\NormalTok{,}\StringTok{"Male"}\NormalTok{),}\DataTypeTok{each=}\DecValTok{20}\NormalTok{),}
                     \DataTypeTok{Sequence=}\KeywordTok{rep}\NormalTok{(}\KeywordTok{rep}\NormalTok{(}\KeywordTok{c}\NormalTok{(}\StringTok{"switch"}\NormalTok{,}\StringTok{"repeat"}\NormalTok{),}\DataTypeTok{each=}\DecValTok{10}\NormalTok{),}\DecValTok{2}\NormalTok{),}
                     \DataTypeTok{RT=}\KeywordTok{c}\NormalTok{(women_switch,women_repeat,}
                          \NormalTok{men_switch,men_repeat))}

\KeywordTok{kable}\NormalTok{(all_data,}\DataTypeTok{format=}\StringTok{"latex"}\NormalTok{)}
\end{Highlighting}
\end{Shaded}

\begin{tabular}{r|l|l|r}
\hline
Subject & Gender & Sequence & RT\\
\hline
1 & Female & switch & 572\\
\hline
2 & Female & switch & 621\\
\hline
3 & Female & switch & 569\\
\hline
4 & Female & switch & 609\\
\hline
5 & Female & switch & 574\\
\hline
6 & Female & switch & 598\\
\hline
7 & Female & switch & 590\\
\hline
8 & Female & switch & 623\\
\hline
9 & Female & switch & 586\\
\hline
10 & Female & switch & 590\\
\hline
1 & Female & repeat & 556\\
\hline
2 & Female & repeat & 566\\
\hline
3 & Female & repeat & 588\\
\hline
4 & Female & repeat & 611\\
\hline
5 & Female & repeat & 575\\
\hline
6 & Female & repeat & 582\\
\hline
7 & Female & repeat & 576\\
\hline
8 & Female & repeat & 578\\
\hline
9 & Female & repeat & 558\\
\hline
10 & Female & repeat & 581\\
\hline
11 & Male & switch & 660\\
\hline
12 & Male & switch & 657\\
\hline
13 & Male & switch & 622\\
\hline
14 & Male & switch & 651\\
\hline
15 & Male & switch & 664\\
\hline
16 & Male & switch & 623\\
\hline
17 & Male & switch & 642\\
\hline
18 & Male & switch & 676\\
\hline
19 & Male & switch & 659\\
\hline
20 & Male & switch & 618\\
\hline
11 & Male & repeat & 585\\
\hline
12 & Male & repeat & 571\\
\hline
13 & Male & repeat & 575\\
\hline
14 & Male & repeat & 594\\
\hline
15 & Male & repeat & 572\\
\hline
16 & Male & repeat & 605\\
\hline
17 & Male & repeat & 556\\
\hline
18 & Male & repeat & 650\\
\hline
19 & Male & repeat & 594\\
\hline
20 & Male & repeat & 606\\
\hline
\end{tabular}

\subsection{plotting the data}\label{plotting-the-data}

We can plot the data at least two ways. See the bar and line graphs
below. Note that the x-axis changes between graphs.

\begin{Shaded}
\begin{Highlighting}[]
\KeywordTok{library}\NormalTok{(ggplot2)}
\KeywordTok{library}\NormalTok{(plyr)}
\NormalTok{sde<-function(x)\{}\KeywordTok{sd}\NormalTok{(x)/}\KeywordTok{length}\NormalTok{(x)\}}
\NormalTok{plot_means<-}\KeywordTok{ddply}\NormalTok{(all_data,.(Gender,Sequence),summarise,}
                       \DataTypeTok{MeanRT=}\KeywordTok{mean}\NormalTok{(RT),}
                       \DataTypeTok{SE=}\KeywordTok{sde}\NormalTok{(RT))}

\NormalTok{limits <-}\StringTok{ }\KeywordTok{aes}\NormalTok{(}\DataTypeTok{ymax =} \NormalTok{MeanRT +}\StringTok{ }\NormalTok{SE, }\DataTypeTok{ymin =} \NormalTok{MeanRT -}\StringTok{ }\NormalTok{SE)}

\KeywordTok{ggplot}\NormalTok{(plot_means,}\KeywordTok{aes}\NormalTok{(}\DataTypeTok{x=}\NormalTok{Gender, }\DataTypeTok{y=}\NormalTok{MeanRT, }\DataTypeTok{group=}\NormalTok{Sequence,}\DataTypeTok{fill=}\NormalTok{Sequence))+}
\StringTok{  }\KeywordTok{geom_bar}\NormalTok{(}\DataTypeTok{position=}\StringTok{"dodge"}\NormalTok{,}\DataTypeTok{stat=}\StringTok{"identity"}\NormalTok{)+}
\StringTok{  }\KeywordTok{geom_errorbar}\NormalTok{(limits, }\DataTypeTok{width=}\NormalTok{.}\DecValTok{3}\NormalTok{,}\DataTypeTok{position=}\KeywordTok{position_dodge}\NormalTok{(.}\DecValTok{9}\NormalTok{))+}
\StringTok{  }\KeywordTok{theme_classic}\NormalTok{(}\DataTypeTok{base_size=}\DecValTok{12}\NormalTok{)+}
\StringTok{  }\KeywordTok{ylab}\NormalTok{(}\StringTok{"Mean RT"}\NormalTok{)+}
\StringTok{  }\KeywordTok{xlab}\NormalTok{(}\StringTok{"Trial sequence"}\NormalTok{)}
\end{Highlighting}
\end{Shaded}

\includegraphics{Lab5_files/figure-latex/unnamed-chunk-2-1}

\begin{Shaded}
\begin{Highlighting}[]
\KeywordTok{ggplot}\NormalTok{(plot_means,}\KeywordTok{aes}\NormalTok{(}\DataTypeTok{x=}\NormalTok{Sequence, }\DataTypeTok{y=}\NormalTok{MeanRT, }\DataTypeTok{group=}\NormalTok{Gender,}\DataTypeTok{shape=}\NormalTok{Gender))+}
\StringTok{  }\KeywordTok{geom_line}\NormalTok{()+}
\StringTok{  }\KeywordTok{geom_point}\NormalTok{()+}
\StringTok{  }\KeywordTok{geom_errorbar}\NormalTok{(limits, }\DataTypeTok{width=}\NormalTok{.}\DecValTok{3}\NormalTok{)+}
\StringTok{  }\KeywordTok{theme_classic}\NormalTok{(}\DataTypeTok{base_size=}\DecValTok{12}\NormalTok{)+}
\StringTok{  }\KeywordTok{ylab}\NormalTok{(}\StringTok{"Mean RT"}\NormalTok{)+}
\StringTok{  }\KeywordTok{xlab}\NormalTok{(}\StringTok{"Trial sequence"}\NormalTok{)}
\end{Highlighting}
\end{Shaded}

\includegraphics{Lab5_files/figure-latex/unnamed-chunk-3-1}

The graphs show that the switch costs (difference between repeat and
switch trials) is smaller for women and men. Which is good, because we
are simulating the data with this outcome in mind.

\subsection{Running the ANOVA}\label{running-the-anova}

The next step is to conduct an ANOVA. This design has two factors or
independent variables, gender and trial sequence. The gender variable is
between-subjects, and the trial sequence variable is within subjects.
So, we will run a 2 (Gender: Female vs.~Male) x 2 (trial sequence:
Repeat vs.~Switch) mixed design ANOVA with Gender as the betwen-subjects
factor, and trial sequence as the within-subjects factor.

\begin{Shaded}
\begin{Highlighting}[]
\KeywordTok{library}\NormalTok{(broom)}
\NormalTok{all_data$Subject<-}\KeywordTok{as.factor}\NormalTok{(all_data$Subject)}
\NormalTok{aov.out<-}\KeywordTok{aov}\NormalTok{(RT~Gender*Sequence+}\KeywordTok{Error}\NormalTok{(Subject/Sequence),all_data)}
\NormalTok{aov_summary<-}\KeywordTok{summary}\NormalTok{(aov.out)}
\KeywordTok{kable}\NormalTok{(}\KeywordTok{xtable}\NormalTok{(aov_summary),}\DataTypeTok{format=}\StringTok{"latex"}\NormalTok{)}
\end{Highlighting}
\end{Shaded}

\begin{tabular}{l|r|r|r|r|r}
\hline
  & Df & Sum Sq & Mean Sq & F value & Pr(>F)\\
\hline
Gender & 1 & 11458.225 & 11458.2250 & 22.10104 & 1.78e-04\\
\hline
Residuals & 18 & 9332.050 & 518.4472 & NA & NA\\
\hline
Sequence & 1 & 13140.625 & 13140.6250 & 38.47507 & 7.50e-06\\
\hline
Gender:Sequence & 1 & 4060.225 & 4060.2250 & 11.88813 & 2.87e-03\\
\hline
Residuals & 18 & 6147.650 & 341.5361 & NA & NA\\
\hline
\end{tabular}

\begin{Shaded}
\begin{Highlighting}[]
\NormalTok{mt<-}\KeywordTok{model.tables}\NormalTok{(aov.out,}\StringTok{"means"}\NormalTok{)}
\NormalTok{mt}
\end{Highlighting}
\end{Shaded}

\begin{verbatim}
## Tables of means
## Grand mean
##         
## 602.075 
## 
##  Gender 
## Gender
## Female   Male 
##  585.1  619.0 
## 
##  Sequence 
## Sequence
## repeat switch 
##  583.9  620.2 
## 
##  Gender:Sequence 
##         Sequence
## Gender   repeat switch
##   Female 577.1  593.2 
##   Male   590.8  647.2
\end{verbatim}

The above shows the ANOVA table and the means for the main effects and
interaction. We also conduct t.test comparisons to look at the switch
costs separately for men and women.

\begin{Shaded}
\begin{Highlighting}[]
\NormalTok{FemaleT<-}\KeywordTok{t.test}\NormalTok{(RT~Sequence,all_data[all_data$Gender==}\StringTok{"Female"}\NormalTok{,],}\DataTypeTok{paired=}\OtherTok{TRUE}\NormalTok{,}\DataTypeTok{var.equal=}\OtherTok{TRUE}\NormalTok{)}
\NormalTok{MaleT<-}\KeywordTok{t.test}\NormalTok{(RT~Sequence,all_data[all_data$Gender==}\StringTok{"Male"}\NormalTok{,],}\DataTypeTok{paired=}\OtherTok{TRUE}\NormalTok{,}\DataTypeTok{var.equal=}\OtherTok{TRUE}\NormalTok{)}
\NormalTok{FemaleT}
\end{Highlighting}
\end{Shaded}

\begin{verbatim}
## 
##  Paired t-test
## 
## data:  RT by Sequence
## t = -2.3034, df = 9, p-value = 0.04674
## alternative hypothesis: true difference in means is not equal to 0
## 95 percent confidence interval:
##  -31.9115634  -0.2884366
## sample estimates:
## mean of the differences 
##                   -16.1
\end{verbatim}

\begin{Shaded}
\begin{Highlighting}[]
\NormalTok{MaleT}
\end{Highlighting}
\end{Shaded}

\begin{verbatim}
## 
##  Paired t-test
## 
## data:  RT by Sequence
## t = -6.0205, df = 9, p-value =
## 0.0001975
## alternative hypothesis: true difference in means is not equal to 0
## 95 percent confidence interval:
##  -77.59196 -35.20804
## sample estimates:
## mean of the differences 
##                   -56.4
\end{verbatim}

\subsection{Writing up the results}\label{writing-up-the-results}

The next step is to interpret the results and write them up. Here is an
example write-up.

The mean reaction times for each subject in trial sequence condition
were submitted to a 2 (Gender: Female vs.~Male) x 2 (trial sequence:
Repeat vs.~Switch) mixed design ANOVA with Gender as the betwen-subjects
factor, and trial sequence as the within-subjects factor. Mean reaction
times in each condition collapsed across subjects are displayed in
Figure 1.

The main effect of gender was significant, F(1, 18) = 22.1, MSE =
518.45, p \textless{} 0. Women (585 ms) had faster mean reaction times
than men (619 ms).

The main effect of trials sequence was significant, F(1, 18) = 38.48,
MSE = 341.54, p \textless{} 0. Repeat trials (584) had faster mean
reaction times than switch trials (620).

Most important was the significant two-way interaction between gender
and trial sequence, F(1, 18) = 11.89, MSE = 341.54, p \textless{} 0.003.
We interpreted the interaction further by conducting the following
comparisons. Women showed a significant switch cost, t(9) = -2.3, p =
0.047, with faster mean reaction times for repeat (577) than switch
(593) trials. Men also showed a significant switch cost, t(9) = -6.02, p
= 0, with faster mean reaction times for repeat (591) than switch trials
(647). The presence of an interaction indicates that the size of the
switch cost for women was significantly smaller than the size of the
switch cost for men.










\chapter{Lab 6: Stroop Mini Project}
\lhead{\allcaps{Lab 6: Stroop Mini Project}}
\openepigraph{The simple act of paying attention can take you a long way.
}{---Keanu Reeves}


In the second mini-project, you will read, summarize and discuss the paper by \citeauthor{raz_suggestion_2006} (2013)\cite{raz_suggestion_2006}. This paper provides some background about the Stroop effect, which is a classic measure of selective attention, and then shows one manipulation that effectively helps people overcome Stroop interference. Your task in this lab will be to replicate the Stroop effect in one condition, and then attempt to change the size of the effect (make larger or smaller) in another condition.

\section{What's in store}

\begin{enumerate}
\item Students read paper and write QALMRI (15-20)
\item	Group discussion about paper (15-20)
\item	Students instructed their task is create their own Stroop design and employ a manipulation that increases or decreases the size of the Stroop effect
\item	Students break into groups and conduct a Stroop experiment, measuring the size of the Stroop effect in a "normal" condition, and in their manipulated condition.
\item	Groups analyze conduct a 2x2 ANOVA to see if their interaction was significant
\end{enumerate}

\section{Some background on the Stroop effect}

The Stroop effect is a well-known and classic phenomena in experimental psychology. The effect was first reported by J. R. Stroop (1935). Several hundreds of Stroop experiments have been conducted since 1935 and these experiments are summarized in Macleod's (1992) review.\cite{Stroop1935,macleod_half_1991}

The original Stroop task involved color naming of word stimuli that are printed in different ink colors. There are two important conditions involving congruent and incongruent item types. \emph{Congruent} items occur when the ink-color of the word matches the name of the word (e.g., the word red printed in red ink: \textcolor{red}{RED}). Incongruent items occur when the ink-color of the word does not match the name of the word (e.g., the word blue printed in red ink: \textcolor{red}{BLUE}). For each of these stimuli the task is to identify the ink-color of the word and ignore the written meaning of the word. Here are a few more examples of congruent and incongruent Stroop items:

Congruent: 	\textcolor{blue}{BLUE}, \textcolor{green}{GREEN}, \textcolor{yellow}{YELLOW}, \textcolor{red}{RED}

Incongruent:	\textcolor{red}{BLUE}, \textcolor{blue}{GREEN}, \textcolor{red}{YELLOW}, \textcolor{green}{RED} 

In the original set of experiments participants were presented with long lists of items and asked to read through the lists naming only the ink-colors and ignoring the word information. The important finding was that people were faster to finish reading lists that were composed of congruent items than incongruent items. This difference in reading time is termed the Stroop effect. For example, let’s say you were given a list of 60 congruent words and you read all of the ink-color names in 71 seconds. Next, you receive a list of 60 incongruent words and it takes you 94 seconds to read all of the ink-color names. You would compute your Stroop effect by taking the difference between the Congruent and Incongruent reading times. Congruent reading times are subtracted from incongruent reading times to give a positive Stroop effect value:

Stroop effect = Incongruent – Congruent  = 94 seconds – 71 seconds =  23 seconds

The Stroop test can be administered in many different ways. A common modern variant of the task is to present a single Stroop item on a computer screen per trial and have participants identify the ink-color by pressing a key or typing out the response. Several trials could be presented over the course of an experimental session, and the experimenter could vary whether upcoming trials are congruent or incongruent. The trial-based version of the Stroop procedure provides more experimental control and more precise measurement of reaction times.  This provides a more fine-grained measure of the time taken to respond congruent and incongruent items. Stroop effects using this procedure are usually measured in the milliseconds (ms). For example, reaction times for congruent items are usually around 500 ms, and reaction times for incongruent items are usually around 600 ms. So, the overall Stroop effect might be around 100 ms. 

\subsection{Stroop and Selective Attention}

The Stroop effect has been used as a tool to study various aspects of learning and attention. For example the Stroop effect has been used as a tool to study cognitive control. Cognitive control refers broadly to the psychological processes that allow people to plan, coordinate, and execute actions necessary to accomplish goals. A central aspect of cognitive control involves the attention processes that are responsible for selecting task-relevant information and ignoring task-irrelevant information when performing a task. 

The Stroop procedure provides a simple, yet effective, method for presenting people with two sources of information, one that is task-relevant (color) and one that is task-irrelevant (word). For this reason, Stroop items are sometimes referred to as bi-valent stimuli, in that they present two sources of information. Congruent Stroop items (blue in blue) do not present much of a selective attention challenge, both the word and the color information point to the same response. Incongruent items (blue in red) do present a selective attention challenge, the color points to the correct response and the word points to an incorrect response. When faced with incongruent items people must pay attention to the relevant color information and ignore or somehow filter out the irrelevant word information. The fact that people get Strooped, or that the Stroop effect exists at all, tells us something important about selective attention in this task. Specifically, people can not completely ignore the irrelevant word information. If people could completely ignore the irrelevant word information then the Stroop effect would cease to appear. People would be just as fast identifying congruent and incongruent items because they would be able to successfully ignore word information. 

Using the logic above the Stroop effect is often taken as a measure of selective attention ability. This means that changes in the size of the Stroop effect may represent differences in selective attention ability. People who have very large Stroop effects have poor attentional control over their ability to ignore the word stimulus. People who have very small Stroop effects have excellent attentional control over their ability to ignore the word stimulus. 






\chapter{Labs 7 and 8: Paper Project 2 }
\lhead{\allcaps{Labs 7 and 8: Paper Project 2}}
\input{Paper2.tex}

\chapter{Labs 9 - 14: The Final Project}
\lhead{\allcaps{Labs 9 - 14: The Final Project}}
\openepigraph{The purpose of psychology is to give us a completely different idea of the things we know best.}{---Paul Valery}

\section{Overview}

Your task for the final project is to design and run your own experiment. There are 3 major components.

\begin{itemize}
\item 2-3 minute individual presentation
\item Individual APA research report
\item 10 minute group presentation
\end{itemize}

For your individual presentation you will have the opportunity to develop your own experimental idea. Following the individual presentations you will be divided into smaller groups, and will decide on a final experiment idea. Each group will be required to have their experimental design approved by the lab instructor. Once the experiment is approved data collection can begin. Your lab instructors will help you with your experimental design to ensure that your project is feasible given the constraints that you are working with. Every student in the group will be responsible for writing their own APA research report on the findings of the experiment conducted by the group. As well as the paper, each group will be responsible for a 10 minute presentation that explains the findings of the research.

\subsection{Requirements}

The following applies for all aspects of the following assignments.

\begin{enumerate}
\item The final project (or proposal for individual presentation) must be a factorial design with two manipulated independent variables.
\item The project must identify an effect from the literature that can be measured by one of the independent variables. For example, the choice of effect could be the Stroop effect, in which case the first independent variable would be congruency (congruent vs. incongruent).
\item The second independent variable will involve a manipulation intended to influence the size of the effect under investigation. The manipulation could be intended to increase the size of the effect relative to an established condition, or to decrease or eliminate the the effect. For example, in a Stroop experiment, the second independent variable could be word-size, and the empirical question could be whether the Stroop effect is larger when the words are printed in a large font, compared to when the words are printed in a smaller font.
\item In other words, the design of the final project will attempt to produce an interaction effect.
\end{enumerate}

\subsection{2-3 minute individual presentation (5\% of total grade)}

Every student is responsible for one 2-3 minute presentation that outlines their proposal for the final project. After the presentation, students will form groups, and the groups will decide on a particular proposal for the final project (from the presentations, or a new proposal)

Each individual presentation at a minimum should accomplish the following:
\begin{itemize}
\item Describe the effect that will be studied
\item Describe the proposed manipulation that will influence the effect
\item Show the predicted results in a graph
\item Explain the hypothesis or theory that would lead you to predict those results
\end{itemize}

Presentations should be made in powerpoint. You are allowed to develop any kind of experiment that interests you. However, your experiment must involve a factorial design. 

\subsection{APA style research report (15\% of total grade)}

Every student will be responsible for reporting the findings of their group experiment in an APA style research report. The paper should have a minimum of 5 pages, and should include an introduction, methods section, results section, discussion, and reference section. The paper should contain at least 3 citations to relevant papers from the psychological literature. The format of the paper will be identical to all of the previous labs. The paper is not a group assignment. Each student will write their own paper.

\subsection{10 minute group presentation (5\% of total grade)}

After data collection and data analysis has taken place, each group will be responsible for presenting their findings. The group presentation should last approximately 10 minutes. The group presentation should be shared amongst the members of the group, with each member responsible for presenting a section of the material. The following material should be covered in the presentation:

\begin{itemize}
\item Explain relevant background knowledge
\item Explain the hypothesis under investigation
\item Explain the proposed design of the experiment
\item Explain the predicted findings of the experiment
\item Explain the actual findings
\item Discuss the meaning of the findings as they relate to the hypotheses under investigation.
\end{itemize}

\chapter{Statistical Analysis And Reporting}
\lhead{\allcaps{Statistical Analysis And Reporting}}
\openepigraph{Memory... is the diary that we all carry about with us.}{---Oscar Wilde}

\section{Excel Tips}

This document gives a brief introduction to using excel for data analysis. The guide covers basic excel operation, formula use, and pivot tables. Many first time users will find aspects of excel confusing, this guide is intended to give you enough basic knowledge to get you started. There are extensive help guides published on the internet, and if you find yourself confronted with a problem that is not discussed here you will often find valuable help by searching Google for the answer. If you have suggestions for excel tips that should be included in this document please let your lab instructor know.

\subsection{What is Excel?}

Excel is a spreadsheet tool. It allows you to organize and analyze data that are stored in column and row format. Excel is extremely powerful. With enough know-how almost all of the data analysis that statistics programs like SPSS are built for, can be accomplished directly in Excel. 


\subsection{What is Excel used for in this course?}


You can use excel (or another spreadsheet program) to:


\begin{itemize} 

\item Store your data, 

\item to perform basic analysis of your data (e.g., getting averages and standard deviations, computing correlations etc.), and 

\item to create figures, graphs, and tables to present your data.
\end{itemize}


\section{Using Excel}

\subsection{The excel window}
 
Excel is a spreadsheet that contains different rows (the numbers), and different columns (the letters).
The individual cells can be used to hold data in the form of numbers or letters, to analyze data, and to report data using tables and figures.

\begin{figure}
      \includegraphics[width=.7\linewidth]{LabmanualFigures/Excel1.pdf}
      \caption{The excel window}
      \label{fig:excelwindow}
\end{figure}

\subsection{Inputting data}

Type or paste individual numbers or letters into individual cells, and then press return.

\begin{figure}
      \includegraphics[width=.5\linewidth]{LabmanualFigures/Excel2.pdf}
      \caption{Inputting data}
      \label{fig:excel2}
\end{figure}

\subsection{Addressing cells}

Excel uses a Letter-Number system to address or point to specific cells. As you can see, all of the numbers in this example have been added to rows 2-9 in Column B. Using the Letter- Number system, the B2=1, B3=5, B4=4, B6=7, and so on.


\subsection{Setting one cell to equal another}

\begin{figure}
      \includegraphics[width=.5\linewidth]{LabmanualFigures/Excel3.pdf}
      \caption{Making one cell equal another}
      \label{fig:excel3}
\end{figure}

If you click on an empty cell, you can make this cell have the same contents as another cell by typing the (=) sign, then clicking on the cell you want to duplicate.
E.g., click on A2, type (=), then click on B2. B2 initially had a 1 in it, now when you press enter, cell A2 will also have a 1 in it. Now if you change the contents of B2 to another number (say 5), the contents of A2 will also change to 5.
Note: this trick becomes handy later on to help you quickly manipulate data in excel

\subsection{Adding cells together, and copying commands across other cells}

\begin{figure}
      \includegraphics[width=.5\linewidth]{LabmanualFigures/Excel4.pdf}
      \caption{Adding}
      \label{fig:excel4}
\end{figure}

Let's say you had two columns of numbers. For each row you want to compute the sum of the first and second number (e.g., the first two numbers are 1 and 1, you want to add them together to get 2.
Click on a new column (C2) and type the equal sign =
Now click on A2 (which contains 1) once with the mouse, and then click on B2 (which also contains 1). Now the formula in C2 will automatically contain =A2+B2
When you press enter the two numbers will be summed, and the answer in C2 with be 2.

\begin{figure}
      \includegraphics[width=.5\linewidth]{LabmanualFigures/Excel5.pdf}
      \caption{Applying across cells}
      \label{fig:excel5}
\end{figure}

If you want to do the same operation for all of the rows, all you have to do is click on C2. Notice there is a little blue square at the bottom right hand corner of the cell.
   

\subsection{Using a formula to add two numbers together}
You can do the same addition operation from above by using the sum formula. In this example you would click on C2, then type =sum(
Then you need to tell excel the range of columns and rows that you want to sum over. In this case you can just click on A2, and drag across to B2. This will make a temporary blue rectangle, which represents the cells that will be summed together
Complete the formula by using the )
Then press enter, and you will get the answer. 

\begin{figure}
      \includegraphics[width=.5\linewidth]{LabmanualFigures/Excel6.pdf}
      \caption{Adding using the sum formula}
      \label{fig:excel6}
\end{figure}

You can also copy this formula down across the other cells in the same way as above
If you click on the little square, then drag down, the formula will be applied to each of the rows.
And, then you have all the answers without having to enter the formula separately for each row.

\begin{figure}
      \includegraphics[width=.5\linewidth]{LabmanualFigures/Excel7.pdf}
      \caption{Completing the formula}
      \label{fig:excel7}
\end{figure}    

\subsection{Getting an average}

Using the same method as the sum formula, you can also compute the mean of a set of numbers by using the average function instead of the sum function. The process is the same. Select a cell, type =average( then select the cells you want (in the example B2:B9) and press enter.

\begin{figure}
      \includegraphics[width=.5\linewidth]{LabmanualFigures/Excel8.pdf}
      \caption{The average formal}
      \label{fig:excel8}
\end{figure}

\subsection{Other formulas}

\begin{itemize}
\item Max – finds the biggest number
\item Min- finds the smallest number
\item Stdev- computes the standard deviation
\item Countif – counts the number of times a specific value occurs
\end{itemize}

Excel has a dictionary of other functions that may be useful, you can look them using help, or using insert: function, from the menu.


\subsection{Selecting a range of cells}
Formulas usually require you to enter a range of cells to compute. The format is put the upperleftmost cell first (e.g., A2), and then the bottomrightmost cell last (e.g., C9). So, A2:C9 would signify the
following rectangle.

\begin{figure}
      \includegraphics[width=.5\linewidth]{LabmanualFigures/Excel9.pdf}
      \caption{selecting a range}
      \label{fig:excel9}
\end{figure}
  

\subsection{Copying a formula to another cell: relative coordinates}

If you were to now select cell A13 it would have the formula =A2:C9 inside. If you copied this cell, and then pasted it into the cell beside it A14, Excel would automatically move the rectangle over one. This is because without further specification, excel always treats cells in relative coordinates.

\begin{figure}
      \includegraphics[width=.5\linewidth]{LabmanualFigures/Excel10.pdf}
      \caption{Relative coordinates}
      \label{fig:excel10}
\end{figure}
 

\subsection{Absolute coordinates}
You can control whether or not excel uses relative coordinates. When you set the range you can insert the \$ sign to make sure that excel holds the rectangle in place.
For example:
A13 =A2:C9
-this formula has no \$s, as in the above example, if you copy this formula to another cell say
B13, then B13=B2:D9, and not the original A2:C9
A13=\$A\$2:\$C\$9
-This formula has \$s infront of both letter and number for each cell in the rectangle. Now, when
the formula is copied to another cell, the original rectangle will be used. E.g., B13=A2:C9 You can set the only columns or the row or both to absolute coordinates using the \$ sign.
 

\subsection{Sorting data}

\begin{figure}
      \includegraphics[width=.5\linewidth]{LabmanualFigures/Excel11.pdf}
      \caption{Some numbers to sort}
      \label{fig:excel11}
\end{figure}
 

If you had a bunch of numbers in random order, you could easily sort them by selecting the column or row of numbers, then click on Data from the menu, and choose sort:

\begin{figure}
      \includegraphics[width=.5\linewidth]{LabmanualFigures/Excel12.pdf}
      \caption{Sorting}
      \label{fig:excel12}
\end{figure}
 


 
You will see a menu something like this. You can choose to sort the current column ascending (smallest to largest) or descending (largest to smallest). Click OK and the data will be rearranged in order.
  

\subsection{Making a histogram}

\begin{figure}
      \includegraphics[width=.7\linewidth]{LabmanualFigures/Excel13.pdf}
      \caption{Making a histogram}
      \label{fig:excel13}
\end{figure}
 


If you want to know how many responses occurred for a particular range of values you can create a histogram. The following example is a very simple way to count individual categories of values in a data set.
Column A has the example data. In column C, I created cells with values ranging from 1 to 7 (Cells C2 : C8). In cell D2, I typed in the countif(range,value) formula. The range refers to the selected data (\$A\$2:\$A\$30). 

Note, there are \$ signs used before the letter and numbers so that the selected data will be the same when the formula is copied. In this case, it is the value 1, which is in Cell C2. The value refers to the entity that is being counted. When you press enter, excel will compute the number of times that 1 appears in the data. You can now drag cell (D2) with the formula down, and it will be used to count the rest of the numbers.
 


\subsection{Making a table}

\begin{figure}
      \includegraphics[width=.7\linewidth]{LabmanualFigures/Excel14.pdf}
      \caption{Making a table}
      \label{fig:excel14}
\end{figure}
 
Now that you have the frequency of each of the values from 1-7.
You can make a graph by selecting the column with the frequencies in it, and then click on INSERT, from the menu, and choose chart. Select a column chart, and then you will see:
You can then edit this chart. You should insert a title for the figure, an x-axis label (e.g., the numbers on the bottom represent different categories from the data), and a y-axis label (the numbers on the left side represent the frequency count).
 

\begin{figure}
      \includegraphics[width=.7\linewidth]{LabmanualFigures/Excel15.pdf}
      \caption{A figure for the histogram}
      \label{fig:excel15}
\end{figure}
 



\subsection{Paired Samples T-test} 
Suppose you ran an experiment with one independent variable that has 2 levels. You used a within- subject design so each participant contributed data to each of the 2 levels. You want to find out if there was significant effect. That is, is the mean performance in Level 1 different from mean performance in Level 2. An appropriate test is the paired-samples t-test.
Here is some sample data in Excel.
 and the number 1 tells excel to use a paired samples t-test. The resulting p-value is <.05 so we now know that there was a significant effect. The means for level 1 were significantly smaller than the means for level 2.

\begin{figure}
      \includegraphics[width=.7\linewidth]{LabmanualFigures/Excel16.pdf}
      \caption{t test in excel}
      \label{fig:excel16}
\end{figure}
 


Tip: Even before you run a paired samples t-test you should have a good idea whether the test will be significant. You can get a ballpark estimate by computing the differences between means for each subject. This has been done in the example under the column labeled Difference. Here, the mean for level 1 has been subtracted from the mean for level 2 for each subject. This gives us a difference score. Look at all of the difference scores. You will see that almost all of them (except for 2) are positive. So, most of the subjects showed the effect. As a ballpark rule, when most of the subjects show the effect you should expect that a t-test will likely show a significant result.

\section{SPSS}

\subsection{Paired Samples T-test in SPSS} 

\begin{figure}
      \includegraphics[width=.5\linewidth]{LabmanualFigures/SPSS1.pdf}
      \caption{Copy the data into SPSS}
      \label{fig:spss1}
\end{figure}

1. Copy the data for each level of the independent variable into separate columns in the data editor. In the example I've given new names (lev1, lev2) to each of the conditions.


\begin{figure}
      \includegraphics[width=.5\linewidth]{LabmanualFigures/SPSS2.pdf}
      \caption{Choose t-test}
      \label{fig:SPSS2}
\end{figure}

2. Next,choose analyze from the menu, select Compare Means, then select Paired-Samples T Test


3. You will see the following menu. Select both of your variables (Lev1, Lev2) then press the arrow button to move them into the paired variables list
   

\begin{figure}
      \includegraphics[width=.5\linewidth]{LabmanualFigures/SPSS3.pdf}
      \caption{Select levels}
      \label{fig:SPSS3}
\end{figure}

4. It should look like this... now click the button for the analysis


5. The first box shows basic descriptive statistics, means, number per cell, standard deviation and standard error of the mean.


\begin{figure}
      \includegraphics[width=\linewidth]{LabmanualFigures/SPSS4.pdf}
      \caption{ttest output}
      \label{fig:SPSS4}
\end{figure}

6. The third box shows the t-value, the associated degrees of freedom (df), and the p- value (Sig. (2-tailed)


7. Here's how you write up your results in one sentence.


8. Means were significantly smaller for level 1 (174) than level 2 (190), t(15) = 2.51, p<.05.
  
\subsection{2 x 2 Repeated Measures ANOVA}
\subsection{From Data to Analysis}

Suppose you ran an experiment with 2 independent variables, each with 2 levels. If this is a within- subject design, each participant will contribute data to each of the 4 levels in the design. The appropriate test is a 2x2 repeated measures ANOVA. We begin by looking at the data in excel.


\begin{figure}
      \includegraphics[width=.7\linewidth]{LabmanualFigures/SPSS5.pdf}
      \caption{Sample data for a 2 x 2 factorial design}
      \label{fig:SPSS5}
\end{figure}

1. All of the four conditions are placed into 4 separate columns. The second IV is nested underneath the first IV.

2. The means and standard deviations can be computed directly in excel for each of the conditions

3. The means are then organized in a table, and a figure can be created so that we can see the pattern of results.

4. Main effects: It looks like there are two main effects. One for IV1: level 1 is smaller than level 2. One for IV2: level 1 is smaller than level 2

5. Interaction:It looks like there is an interaction. The difference between L1 and L2 of IV1 is larger for level 1 of IV2 (in red) than level 2 of IV2 (in blue).

6. We need to run a 2x2 repeated
measures ANOVA to determine whether the main effects are significant, and to determine if the interaction is significant.


\subsection{2x2 Repeated measures ANOVA in SPSS}
 
\begin{figure}
      \includegraphics[width=.6\linewidth]{LabmanualFigures/SPSS6.pdf}
      \caption{Copy data into SPSS}
      \label{fig:SPSS6}
\end{figure}

1. Copy the data into SPSS. Make sure each column represents a different condition in the design.

2. Give names to each of the variables. The first column represents IV1 Level1 \& IV2 Level 1. The second column represents IV1 Level 1 \& IV2 Level 2. The third column represents IV1 Level 2 \& IV2 Level 1. The fourth column represents IV1 Level 2 \& IV2 Level 2.

\begin{figure}
      \includegraphics[width=.5\linewidth]{LabmanualFigures/SPSS7.pdf}
      \caption{Choose the model}
 	\label{fig:sp7}
\end{figure}

3. Choose Analyze, General Linear Model, Repeated Measures from the menu.
   
\begin{figure}
      \includegraphics[width=.5\linewidth]{LabmanualFigures/SPSS8.pdf}
      \caption{Name variables}
      \label{fig:SPSS8}
\end{figure}

\begin{figure}
      \includegraphics[width=.5\linewidth]{LabmanualFigures/SPSS9.pdf}
      \caption{ thing a}
      \label{fig:SPSS9}
\end{figure}

4. Name IV1, specify 2 levels

5. Name IV1, specify 2 levels

6. Name dependent variable, click define

7. Select all four conditions, press the first right arrow, if you want more options see below, otherwise press ok.

\begin{figure}
      \includegraphics[width=\linewidth]{LabmanualFigures/SPSS10.pdf}
      \caption{Select conditions}
      \label{fig:SPSS10}
\end{figure}

8. If you pressed options, then you can ask SPSS to report descriptive statistics for each condition. Choose the factors that you want, then press the arrow button to move them into the display means field. Make sure you click descriptive statistics. Then click continue.

\begin{figure}
      \includegraphics[width=\linewidth]{LabmanualFigures/SPSS11.pdf}
      \caption{Option to report descriptive statistics}
      \label{fig:SPSS11}
\end{figure}

\subsection{Finding numbers in the SPSS analysis output}

SPSS gives you lots of information, you need to know what you are looking for. When you report the results from a 2x2 ANOVA, you will have 2 main effects, and 1 interaction. This means you will be looking for 3 F-values, 3 MSEs (Mean squared error terms), and 3 associated p-values. You will also need to know the means for the main effects and the means for the interaction.

\begin{figure}
      \includegraphics[width=\linewidth]{LabmanualFigures/SPSS12.pdf}
      \caption{Descriptive Statistics}
      \label{fig:SPSS12}
\end{figure}

If you chose the option for descriptive stats, then you will see a small table with Means and standard deviations for each condition in the design. You will use these means to compute averages for the main effects. You will use these means to report the pattern for the interaction.

\subsection{The ANOVA table}
All of the information that you need to report main effects and interactions is available in  the "tests of within-subjects effects" table.

\begin{figure}
      \includegraphics[width=\linewidth]{LabmanualFigures/SPSS13.pdf}
      \caption{ANOVA table}
      \label{fig:SPSS13}
\end{figure}

\subsection{Main effect for IV 1}

Each main effect is listed under source. Each main effect has a corresponding error term listed below. The source for the main effect of independent variable 1 is labeled IV one. You will see it has a corresponding df of 1, an F-value of 112.371, and a p-value <.05. This main effect is significant. You would report this in the following way.

The main effect for independent variable 1 was significant, F(1,15) = 112.371, MSE = 656.47, p<.05. 

The 15 comes from the df from the error term for IVone. As well, the MSE (656.47) comes from the Mean Square from the error term. When you report the main effect, you will also need to report the pattern. The above sentence simply tells the reader that there was a significant difference between the levels of IVone, however it does not explain whether level 1 was larger or smaller than level 2. To report the main effect you will need to compute the averages for each level of IVone. You can do this directly in excel, or you can have SPSS compute these numbers for you by checking the appropriate options before running the analysis. You will need all of the numbers from the above descriptive statistics table. 

First, we find the average for level 1 of IVone (477.5 + 499)/2 = 488

Second, we find the average for level 2 of IVone (659.1875 + 746.5)/2 = 703

Now we can write a complete description of the main effect for IVone. 

The main effect for independent variable 1 was significant, F(1,15) = 112.371, MSE = 656.47, p<.05. Mean performance was lower for level 1 (488) than level 2 (703). 


\subsection{Main effect for IV 2}

Following the same steps as above we look for IV two in the source table and find the dfs, the F-value, the MSE for the error term, and the p-value. 

The main effect for independent variable 2 was significant, F(1,15) = 124.92, MSE = 379.14, p<.05. Mean performance was lower for level 1 (568) than level 2 (623).

\subsection{The interaction effect}

Reporting the interaction uses the same information that you would use to report the main effect. The interaction is listed in the source table as IVone*IVtwo. It has corresponding dfs, F-value, MSE for the error term, and a p-value. 

The interaction between independent variable one and two was significant, F(1,15) = 36.89, MSE = 469.64, p<.05. The difference between level one and level two for independent variable two was smaller for level one (499 - 478 = 21) than level two (747 - 659 = 88) of independent variable 1. 

\subsection{Post Hoc Tests to interpret the interaction}

Post hoc tests are used to clarify the nature of the interaction. In the above example we found a significant interaction. The difference between level 1 and level 2 for IV two was 21 and 88 for level 1 and level 2 of the first independent variable. The significant interaction tells us that 21 and 88 are significantly different from each other. However, we still do not know whether each comparison is significant in an of itself. For example it may be the case that the difference between level 1 and level 2 for IV 2 was only significant for the 2nd level of IV1 (88), and not significant for the 1st level of IV1 (21). In this example you can run a paired t-test, or a one-way repeated measures ANOVA on the specific comparison that you are interested in. For example, you would compare the means from IV1L1/IV2L1 and IV1L1/IV2L2. This comparison would test whether the difference of 21 that was found was actually significant from 0. You will report the statistics for the post-hoc tests in the same manner as you would report t-tests, or F-statistics from the above examples. 

\chapter{Writing An APA Style Paper}
\lhead{\allcaps{Writing an APA style Paper}}
\openepigraph{Memory... is the diary that we all carry about with us.}{---Oscar Wilde}

\section{How to write a research report}

Bean, J. C. (2001). Engaging ideas: The professor’s guide to integrating writing, critical thinking, and active learning in the classroom. San Francisco: Jossey-Bass Publishers

A formal scientific research report is a piece of professional writing addressed to other professionals who are interested in the investigation you conducted. They will want to know why you did the investigation, how you did it, what you found out, and whether your findings were significant and useful. Research reports usually follow a standard five-part format: (1) introduction, (2) methods, (3) results (4) discussion of results, and (5) conclusions and recommendations.

\subsection{Introduction} 
Here you explain briefly the purpose of your investigation. What problem did you address? Why did you address it? You will need to provide enough background to enable the reader to understand the problem being investigated. Sometimes the introduction also includes a “literature review” summarizing previous research addressing the same or a related problem. In many scientific disciplines, it is also conventional to present a hypothesis—a tentative “answer” to the question that your investigation will confirm or disconfirm.

\subsection{Methods} 
This is a “cookbook” section detailing how you did your investigation. It provides enough details so that other researchers could replicate your investigation. Usually, this section includes the following subsections: (a) research design, (b) apparatus and materials, and (c) procedures followed.

\subsection{Results} 
This section sometimes headed “Findings,” presents the empirical results of your investigation. Often, your findings are displayed in figures, tables, graphs, or charts that are referenced in the text. Even though the data are displayed in visuals, the text itself should also describe the most significant data. (Imagine that the figures are displayed on a view graph and that you are explaining them orally, using a pointer. Your written text should transcribe what you would say orally.) Your figures and tables must have sufficient information to stand along, including accurate titles and clear labels for all meaning-carrying features.

\subsection{Discussion of results} 
This is the main part of the report, the part that will be read with the most care by other professionals. Here you explain the significance of your findings by relating what you discovered to the problem you set out to investigate in your introduction. Did your investigation accomplish your purpose? Did it answer your questions? Did it confirm or disconfirm your hypothesis? Are you results useful? Why or why not? Did you discover information that you hadn’t anticipated? Was your research design appropriate? Did your investigation raise new questions? Are there implications from your results that need to be explored? The key to success in this section is to link your findings to the questions and problems raised in the introduction.

\subsection{Conclusions and recommendations} 
In this last section, you focus on the main things you learned from the investigation and, in some cases, on the practical applications of your investigation. If your investigation was a pure research project, this section can be a summary of your most important findings along with recommendations for further research. If your investigation was aimed at making a practical decision (for example, an engineering design decisions), here you recommend appropriate actions. What you say in this section depends on the context of your investigation and the expectations of your readers.

\section{APA style}

The next sections provide information about APA style, and an example research report written in APA style.

\subsection{APA style formatting}

\includepdf[pages=1-6]{APA1.pdf}

\subsection{Sample APA report}

\includepdf[pages=1-8]{APA2.pdf}




\bibliography{mybib}
\bibliographystyle{apalike}

\printindex
\end{document}
