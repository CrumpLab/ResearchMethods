\openepigraph{The purpose of psychology is to give us a completely different idea of the things we know best.}{---Paul Valery}


\section{Overview}

In this project you will attempt to replicate a classic study on face perception (Yin, 1969)\cite{yin_looking_1969}. A replication of the experiment has already been designed. To collect data, you will first participate as a subject in the experiment. Then, as a class you will be introduced to the published paper. You will read the paper and it in class. Then, the class will analyze the collected data to determine whether or not the major effects of interest have been replicated. Data will be collected using pen and paper methods, and analyzed by computer software. Each student will write a 5+ page, APA style report on the project.

\subsection{Things you will learn:}

\begin{itemize}
\item Reading and citing primary source material
\item Writing a brief APA style research report
\item Conducting and reporting Factorial designs
\end{itemize}

\subsection{Background readings:}

Available on the lab website, or download from Brooklyn College library 

\subsection{Grade}
\begin{itemize}
\item  10\% of final grade
\item Graded by your lab instructor. Lab instructor sets due date, and determines whether revised drafts are submittable.
\item If you submit completed versions of all paper assignments (1, 2, and 3), then you get to drop your lowest paper grade from paper 1 or 2. Specifically, you will receive your highest grade from paper 1 or 2 for both papers, thereby eliminating the lower grade. This allows room to learn and improve as you go. 
\end{itemize}


\section{Writing the paper}

There are many resources for help on writing an APA style research report in the lab manual, on the textbook, and the website. Check them out. As well, here is a rough roadmap for writing paper 2.

\subsection{APA formatting, Title and Abstract}

\begin{itemize}
\item Use APA formatting rules.

\item Create a suitable title for the paper

\item Write the abstract : No more than 250 words. The aim is to briefly describe the issue at hand, the experiment, and the results.
\end{itemize}

\subsection{Introduction (around 2 double-spaced pages)}

The goal of the introduction is to first put the research into a broader context, and then narrow the focus to describe the specific research aims.

\begin{enumerate}
\item A. Opening section: (starting broad)

\begin{itemize}
\item about 1 paragraph
\item Discuss a real-world example of the general phenomena under investigation by the paper
\item Tell the reader that the purpose of the current experiment is to conduct a replication of the work in question
\item  Establish the domain and big questions.
\end{itemize}

\item Middle section: Prior work

\begin{itemize}
\item Discuss some examples of previous research that are similar to the present research. You have an opportunity here to look this kind of research up on Google Scholar. One or two examples ought to be enough.
\item Explain the specific question that is being asked in this replication work.
\end{itemize}

\item Final section: (briefly explain the present aims, the experiment and what you expect to find)
\begin{itemize}
\item Explain the hypotheses (alternatives)
\item Explain the logic of how the hypotheses will be tested
\item Briefly explain what the participants will be doing in the task
\item Briefly give predictions for performance in each condition
\end{itemize}

\end{enumerate}


\subsection{Methods (about 1 page)}

The methods section should be a complete recipe that anyone could follow to replicate your experiment. There are lots of details that you can include, some of these are listed below. Be brief and concise

\begin{enumerate}
\item Participants
	\begin{itemize}
	\item how many people?
	\item where did they come from?
	\end{itemize}
\item Materials
	\begin{itemize}
	\item what were the stimuli?
	\item how were they organized?
	\end{itemize}

\item Design \& Procedure
	\begin{itemize}
	\item What was the design
	\item What were the independent variable(s)
	\item What was the dependent variable
	\item Within or between subjects?
	\item How were the stimuli for each trial chosen
	\item Describe the steps each participant took to complete the experiment
	\end{itemize}

\end{enumerate}

\subsection{Results}

The result section is used to report the patterns in the data, and the statistical support for those patterns. You will compute the results using SPSS in the lab computers. The lab manual can be consulted for help on running statistical tests, and for reporting results.

\begin{itemize}
\item Describe the statistical analysis
\item Tell the reader where they can see the data. E.g., the results of experiment 1 are presented in table 1, or in figure 1
\item Make a table or figure to display the data in your paper
\item Report the statistical test, and the pattern of the means.
\end{itemize}

\subsection{Discussion}

The discussion can be used to briefly restate verbally the pattern of the most important results, and then to relate the results to theory and ideas developed in the introduction

\begin{itemize}
\item Highlight the main findings from the experiment
\item Discuss how the data can be explained by the hypothesis. What inferences do you make about the hypotheses based on the research findings?
\item Broaden your discussion. Can the findings be explained by an alternative theory? What can be generalized to the real world? Are there important confounds that prevent us from interpreting our results?
\end{itemize}

\subsection{References and Figures or Tables}

\begin{itemize}
\item Include citations used in the paper using APA style format
\item Include a figure or table to show the results
\end{itemize}

%
\section{Data-analysis tips}\label{lab-1-data-analysis-tips}

In lab one you will be collecting measurements on several dependent
variables, in each of two manipulated conditions (the independent
variable). For each dependent variable you will want to determine
whether the manipulation had an effect. That is, did the independent
variable cause a change in the dependent variable. We know that
differences can sometimes be observed by chance alone, so we want to
conduct an inferential statistical test to determine the probability
that our observed difference could have been produced by chance alone.
To do this we will be conducting several t-tests. This is a short primer
on the process. You can conduct t-tests in the software of your choice,
or by hand using a calculator (or in excel). Here, we will use the free
and open-source statistical package called R, to illustrate the process.

Let's imagine we have two groups of 10 subjects each. Group A receives
condition 1 of the independent variable, and Group B recieves condition
2 of the independent variable. We then measure some behavior for all of
the subject in all of the groups. To make this more concrete, let's say
10 subjects drink coffee, and the the 10 subjects drink tea. Then we
present all of the subjects with a piece of art and ask them rate how
beautiful they think it is on a scale from 1 to 7.

When we collect all the data we should have 20 total ratings, one for
each subject in each group.

For example, if you put the data in a table it might look something like
the following. Note, the grey text box shows the R code used to simulate
the data. For, each group, we sample 10 numbers from a normal
distribution with a mean of 4, and a standard deviation of .5. Then we
put the numbers in a table.

\begin{Shaded}
\begin{Highlighting}[]
\NormalTok{coffee<-}\KeywordTok{round}\NormalTok{(}\KeywordTok{rnorm}\NormalTok{(}\DecValTok{10}\NormalTok{,}\DecValTok{4}\NormalTok{,.}\DecValTok{5}\NormalTok{))}
\NormalTok{tea<-}\KeywordTok{round}\NormalTok{(}\KeywordTok{rnorm}\NormalTok{(}\DecValTok{10}\NormalTok{,}\DecValTok{4}\NormalTok{,.}\DecValTok{5}\NormalTok{))}
\NormalTok{all_data<-}\KeywordTok{data.frame}\NormalTok{(coffee,tea)}
\KeywordTok{kable}\NormalTok{(all_data,}\DataTypeTok{format=}\StringTok{"latex"}\NormalTok{)}
\end{Highlighting}
\end{Shaded}

\begin{tabular}{r|r}
\hline
coffee & tea\\
\hline
4 & 5\\
\hline
3 & 4\\
\hline
3 & 4\\
\hline
3 & 3\\
\hline
4 & 3\\
\hline
4 & 5\\
\hline
4 & 4\\
\hline
4 & 4\\
\hline
4 & 4\\
\hline
3 & 4\\
\hline
\end{tabular}

We can do some quick descriptive statistics, for example, we might want
to know the means of the beauty ratings for the coffee and tea groups.

\begin{Shaded}
\begin{Highlighting}[]
\KeywordTok{mean}\NormalTok{(coffee)}
\end{Highlighting}
\end{Shaded}

\begin{verbatim}
## [1] 3.6
\end{verbatim}

\begin{Shaded}
\begin{Highlighting}[]
\KeywordTok{mean}\NormalTok{(tea)}
\end{Highlighting}
\end{Shaded}

\begin{verbatim}
## [1] 4
\end{verbatim}

The means aren't very different, and of course we should expect they
should be similar. After all, we sampled these means from the exact same
distribution. So, we should expect that on average, the means should
both be close to 4. However, they won't necessarilly be exactly 4,
because of variability introduced by random sampling.

\section{The t-test}\label{the-t-test}

What we want to do next is conduct an independent samples t-test. We
want to determine whether any possible difference between the coffee and
tea groups could have been produced by chance alone. We can conduct a
t-test in R very easily using the t.test function.

\begin{Shaded}
\begin{Highlighting}[]
\KeywordTok{t.test}\NormalTok{(coffee,tea,}\DataTypeTok{var.equal=}\OtherTok{TRUE}\NormalTok{)}
\end{Highlighting}
\end{Shaded}

\begin{verbatim}
## 
##  Two Sample t-test
## 
## data:  coffee and tea
## t = -1.5, df = 18, p-value = 0.151
## alternative hypothesis: true difference in means is not equal to 0
## 95 percent confidence interval:
##  -0.9602459  0.1602459
## sample estimates:
## mean of x mean of y 
##       3.6       4.0
\end{verbatim}

R gives us back the t values, the degrees of freedom (df), and the
associated p-value. The p-value tells us the likelihood that our
difference, or a difference greater than the one we observed could have
been produced by chance.

\section{One more time}\label{one-more-time}

Let's try this whole process again, but this time we will simulate data
with an actual difference between the groups. For example, let's say we
want to simulate the idea that drinking coffee makes people think the
art is less beautiful by at least 2 points, and then reconduct the
t-test with the new simulated data. We will sample numbers from a normal
distribution with mean 3 for the coffee group, and mean 5 for the tea
group (for an average expected difference of 2).

\begin{Shaded}
\begin{Highlighting}[]
\NormalTok{coffee<-}\KeywordTok{round}\NormalTok{(}\KeywordTok{rnorm}\NormalTok{(}\DecValTok{10}\NormalTok{,}\DecValTok{3}\NormalTok{,.}\DecValTok{5}\NormalTok{))}
\NormalTok{tea<-}\KeywordTok{round}\NormalTok{(}\KeywordTok{rnorm}\NormalTok{(}\DecValTok{10}\NormalTok{,}\DecValTok{5}\NormalTok{,.}\DecValTok{5}\NormalTok{))}
\NormalTok{all_data<-}\KeywordTok{data.frame}\NormalTok{(coffee,tea)}
\KeywordTok{kable}\NormalTok{(all_data,}\DataTypeTok{format=}\StringTok{"latex"}\NormalTok{)}
\end{Highlighting}
\end{Shaded}

\begin{tabular}{r|r}
\hline
coffee & tea\\
\hline
3 & 5\\
\hline
3 & 5\\
\hline
3 & 4\\
\hline
3 & 6\\
\hline
2 & 5\\
\hline
4 & 5\\
\hline
3 & 6\\
\hline
3 & 5\\
\hline
3 & 5\\
\hline
3 & 5\\
\hline
\end{tabular}

\begin{Shaded}
\begin{Highlighting}[]
\KeywordTok{mean}\NormalTok{(coffee)}
\end{Highlighting}
\end{Shaded}

\begin{verbatim}
## [1] 3
\end{verbatim}

\begin{Shaded}
\begin{Highlighting}[]
\KeywordTok{mean}\NormalTok{(tea)}
\end{Highlighting}
\end{Shaded}

\begin{verbatim}
## [1] 5.1
\end{verbatim}

\begin{Shaded}
\begin{Highlighting}[]
\KeywordTok{t.test}\NormalTok{(coffee,tea,}\DataTypeTok{var.equal=}\OtherTok{TRUE}\NormalTok{)}
\end{Highlighting}
\end{Shaded}

\begin{verbatim}
## 
##  Two Sample t-test
## 
## data:  coffee and tea
## t = -9, df = 18, p-value = 4.404e-08
## alternative hypothesis: true difference in means is not equal to 0
## 95 percent confidence interval:
##  -2.590215 -1.609785
## sample estimates:
## mean of x mean of y 
##       3.0       5.1
\end{verbatim}

\section{Writing up the results of a
t-test}\label{writing-up-the-results-of-a-t-test}

We've now conducted two different t-tests, and received different
results on each them. You will likely find different results for all of
the t-tests that you conduct for the lab experiment. However, you will
use the basic sentence structure to report all of the results. When you
report the results of your experiment along with statistical tests there
are two important features to include, the pattern of the results, and
the inferential statistic. In this situation, we would simply report the
means and the t-test information. Here are is an with made-up numbers.

The coffee group gave a lower mean beauty rating (M = 3.4) than the tea
group (M = 5.6), and the difference was significant, t (18) = 5.4, p
\textless{} .001.

So, just in one sentence we tell the reader what the means were in both
conditions, as whether the result was significant. APA style recommends
reporting exact p-values when they are greater than .001 (for example p
= .047). If the p-value is less than .001, then you just need to report
p \textless{} .001.





