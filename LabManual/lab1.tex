
\section{Data-analysis tips}\label{lab-1-data-analysis-tips}

In lab one you will be collecting measurements on several dependent
variables, in each of two manipulated conditions (the independent
variable). For each dependent variable you will want to determine
whether the manipulation had an effect. That is, did the independent
variable cause a change in the dependent variable. We know that
differences can sometimes be observed by chance alone, so we want to
conduct an inferential statistical test to determine the probability
that our observed difference could have been produced by chance alone.
To do this we will be conducting several t-tests. This is a short primer
on the process. You can conduct t-tests in the software of your choice,
or by hand using a calculator (or in excel). Here, we will use the free
and open-source statistical package called R, to illustrate the process.

Let's imagine we have two groups of 10 subjects each. Group A receives
condition 1 of the independent variable, and Group B recieves condition
2 of the independent variable. We then measure some behavior for all of
the subject in all of the groups. To make this more concrete, let's say
10 subjects drink coffee, and the the 10 subjects drink tea. Then we
present all of the subjects with a piece of art and ask them rate how
beautiful they think it is on a scale from 1 to 7.

When we collect all the data we should have 20 total ratings, one for
each subject in each group.

For example, if you put the data in a table it might look something like
the following. Note, the grey text box shows the R code used to simulate
the data. For, each group, we sample 10 numbers from a normal
distribution with a mean of 4, and a standard deviation of .5. Then we
put the numbers in a table.

\begin{Shaded}
\begin{Highlighting}[]
\NormalTok{coffee<-}\KeywordTok{round}\NormalTok{(}\KeywordTok{rnorm}\NormalTok{(}\DecValTok{10}\NormalTok{,}\DecValTok{4}\NormalTok{,.}\DecValTok{5}\NormalTok{))}
\NormalTok{tea<-}\KeywordTok{round}\NormalTok{(}\KeywordTok{rnorm}\NormalTok{(}\DecValTok{10}\NormalTok{,}\DecValTok{4}\NormalTok{,.}\DecValTok{5}\NormalTok{))}
\NormalTok{all_data<-}\KeywordTok{data.frame}\NormalTok{(coffee,tea)}
\KeywordTok{kable}\NormalTok{(all_data,}\DataTypeTok{format=}\StringTok{"latex"}\NormalTok{)}
\end{Highlighting}
\end{Shaded}

\begin{tabular}{r|r}
\hline
coffee & tea\\
\hline
4 & 5\\
\hline
3 & 4\\
\hline
3 & 4\\
\hline
3 & 3\\
\hline
4 & 3\\
\hline
4 & 5\\
\hline
4 & 4\\
\hline
4 & 4\\
\hline
4 & 4\\
\hline
3 & 4\\
\hline
\end{tabular}

We can do some quick descriptive statistics, for example, we might want
to know the means of the beauty ratings for the coffee and tea groups.

\begin{Shaded}
\begin{Highlighting}[]
\KeywordTok{mean}\NormalTok{(coffee)}
\end{Highlighting}
\end{Shaded}

\begin{verbatim}
## [1] 3.6
\end{verbatim}

\begin{Shaded}
\begin{Highlighting}[]
\KeywordTok{mean}\NormalTok{(tea)}
\end{Highlighting}
\end{Shaded}

\begin{verbatim}
## [1] 4
\end{verbatim}

The means aren't very different, and of course we should expect they
should be similar. After all, we sampled these means from the exact same
distribution. So, we should expect that on average, the means should
both be close to 4. However, they won't necessarilly be exactly 4,
because of variability introduced by random sampling.

\section{The t-test}\label{the-t-test}

What we want to do next is conduct an independent samples t-test. We
want to determine whether any possible difference between the coffee and
tea groups could have been produced by chance alone. We can conduct a
t-test in R very easily using the t.test function.

\begin{Shaded}
\begin{Highlighting}[]
\KeywordTok{t.test}\NormalTok{(coffee,tea,}\DataTypeTok{var.equal=}\OtherTok{TRUE}\NormalTok{)}
\end{Highlighting}
\end{Shaded}

\begin{verbatim}
## 
##  Two Sample t-test
## 
## data:  coffee and tea
## t = -1.5, df = 18, p-value = 0.151
## alternative hypothesis: true difference in means is not equal to 0
## 95 percent confidence interval:
##  -0.9602459  0.1602459
## sample estimates:
## mean of x mean of y 
##       3.6       4.0
\end{verbatim}

R gives us back the t values, the degrees of freedom (df), and the
associated p-value. The p-value tells us the likelihood that our
difference, or a difference greater than the one we observed could have
been produced by chance.

\section{One more time}\label{one-more-time}

Let's try this whole process again, but this time we will simulate data
with an actual difference between the groups. For example, let's say we
want to simulate the idea that drinking coffee makes people think the
art is less beautiful by at least 2 points, and then reconduct the
t-test with the new simulated data. We will sample numbers from a normal
distribution with mean 3 for the coffee group, and mean 5 for the tea
group (for an average expected difference of 2).

\begin{Shaded}
\begin{Highlighting}[]
\NormalTok{coffee<-}\KeywordTok{round}\NormalTok{(}\KeywordTok{rnorm}\NormalTok{(}\DecValTok{10}\NormalTok{,}\DecValTok{3}\NormalTok{,.}\DecValTok{5}\NormalTok{))}
\NormalTok{tea<-}\KeywordTok{round}\NormalTok{(}\KeywordTok{rnorm}\NormalTok{(}\DecValTok{10}\NormalTok{,}\DecValTok{5}\NormalTok{,.}\DecValTok{5}\NormalTok{))}
\NormalTok{all_data<-}\KeywordTok{data.frame}\NormalTok{(coffee,tea)}
\KeywordTok{kable}\NormalTok{(all_data,}\DataTypeTok{format=}\StringTok{"latex"}\NormalTok{)}
\end{Highlighting}
\end{Shaded}

\begin{tabular}{r|r}
\hline
coffee & tea\\
\hline
3 & 5\\
\hline
3 & 5\\
\hline
3 & 4\\
\hline
3 & 6\\
\hline
2 & 5\\
\hline
4 & 5\\
\hline
3 & 6\\
\hline
3 & 5\\
\hline
3 & 5\\
\hline
3 & 5\\
\hline
\end{tabular}

\begin{Shaded}
\begin{Highlighting}[]
\KeywordTok{mean}\NormalTok{(coffee)}
\end{Highlighting}
\end{Shaded}

\begin{verbatim}
## [1] 3
\end{verbatim}

\begin{Shaded}
\begin{Highlighting}[]
\KeywordTok{mean}\NormalTok{(tea)}
\end{Highlighting}
\end{Shaded}

\begin{verbatim}
## [1] 5.1
\end{verbatim}

\begin{Shaded}
\begin{Highlighting}[]
\KeywordTok{t.test}\NormalTok{(coffee,tea,}\DataTypeTok{var.equal=}\OtherTok{TRUE}\NormalTok{)}
\end{Highlighting}
\end{Shaded}

\begin{verbatim}
## 
##  Two Sample t-test
## 
## data:  coffee and tea
## t = -9, df = 18, p-value = 4.404e-08
## alternative hypothesis: true difference in means is not equal to 0
## 95 percent confidence interval:
##  -2.590215 -1.609785
## sample estimates:
## mean of x mean of y 
##       3.0       5.1
\end{verbatim}

\section{Writing up the results of a
t-test}\label{writing-up-the-results-of-a-t-test}

We've now conducted two different t-tests, and received different
results on each them. You will likely find different results for all of
the t-tests that you conduct for the lab experiment. However, you will
use the basic sentence structure to report all of the results. When you
report the results of your experiment along with statistical tests there
are two important features to include, the pattern of the results, and
the inferential statistic. In this situation, we would simply report the
means and the t-test information. Here are is an with made-up numbers.

The coffee group gave a lower mean beauty rating (M = 3.4) than the tea
group (M = 5.6), and the difference was significant, t (18) = 5.4, p
\textless{} .001.

So, just in one sentence we tell the reader what the means were in both
conditions, as whether the result was significant. APA style recommends
reporting exact p-values when they are greater than .001 (for example p
= .047). If the p-value is less than .001, then you just need to report
p \textless{} .001.

